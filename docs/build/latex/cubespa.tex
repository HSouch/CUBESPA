%% Generated by Sphinx.
\def\sphinxdocclass{report}
\documentclass[letterpaper,10pt,english]{sphinxmanual}
\ifdefined\pdfpxdimen
   \let\sphinxpxdimen\pdfpxdimen\else\newdimen\sphinxpxdimen
\fi \sphinxpxdimen=.75bp\relax
\ifdefined\pdfimageresolution
    \pdfimageresolution= \numexpr \dimexpr1in\relax/\sphinxpxdimen\relax
\fi
%% let collapsible pdf bookmarks panel have high depth per default
\PassOptionsToPackage{bookmarksdepth=5}{hyperref}

\PassOptionsToPackage{warn}{textcomp}
\usepackage[utf8]{inputenc}
\ifdefined\DeclareUnicodeCharacter
% support both utf8 and utf8x syntaxes
  \ifdefined\DeclareUnicodeCharacterAsOptional
    \def\sphinxDUC#1{\DeclareUnicodeCharacter{"#1}}
  \else
    \let\sphinxDUC\DeclareUnicodeCharacter
  \fi
  \sphinxDUC{00A0}{\nobreakspace}
  \sphinxDUC{2500}{\sphinxunichar{2500}}
  \sphinxDUC{2502}{\sphinxunichar{2502}}
  \sphinxDUC{2514}{\sphinxunichar{2514}}
  \sphinxDUC{251C}{\sphinxunichar{251C}}
  \sphinxDUC{2572}{\textbackslash}
\fi
\usepackage{cmap}
\usepackage[T1]{fontenc}
\usepackage{amsmath,amssymb,amstext}
\usepackage{babel}



\usepackage{tgtermes}
\usepackage{tgheros}
\renewcommand{\ttdefault}{txtt}



\usepackage[Bjarne]{fncychap}
\usepackage{sphinx}

\fvset{fontsize=auto}
\usepackage{geometry}


% Include hyperref last.
\usepackage{hyperref}
% Fix anchor placement for figures with captions.
\usepackage{hypcap}% it must be loaded after hyperref.
% Set up styles of URL: it should be placed after hyperref.
\urlstyle{same}

\addto\captionsenglish{\renewcommand{\contentsname}{Contents:}}

\usepackage{sphinxmessages}
\setcounter{tocdepth}{1}



\title{CubeSPA}
\date{Jan 19, 2024}
\release{}
\author{Harrison Souchereau}
\newcommand{\sphinxlogo}{\vbox{}}
\renewcommand{\releasename}{}
\makeindex
\begin{document}

\ifdefined\shorthandoff
  \ifnum\catcode`\=\string=\active\shorthandoff{=}\fi
  \ifnum\catcode`\"=\active\shorthandoff{"}\fi
\fi

\pagestyle{empty}
\sphinxmaketitle
\pagestyle{plain}
\sphinxtableofcontents
\pagestyle{normal}
\phantomsection\label{\detokenize{index::doc}}


\sphinxAtStartPar
\sphinxstylestrong{CubeSPA} is a fully\sphinxhyphen{}featured set of utilities to process, analyze, and display useful information of datacubes, particularly
of astronomical data. It fully supports utilities such as moment map creation, spectral analysis, position\sphinxhyphen{}velocity diagrams, and
many others.

\sphinxAtStartPar
To use \sphinxstylestrong{CubeSPA}, install it using

\sphinxAtStartPar
\sphinxcode{\sphinxupquote{pip install cubespa}} (not yet, but soon)

\sphinxAtStartPar
To begin using CubeSPA, create a \sphinxcode{\sphinxupquote{cubespa.CubeSPA}} object with the filename for your cube in the following way:

\begin{sphinxVerbatim}[commandchars=\\\{\}]
\PYG{g+gp}{\PYGZgt{}\PYGZgt{}\PYGZgt{} }\PYG{n}{filename} \PYG{o}{=} \PYG{l+s+s2}{\PYGZdq{}}\PYG{l+s+s2}{path/to/cube.fits}\PYG{l+s+s2}{\PYGZdq{}}
\PYG{g+gp}{\PYGZgt{}\PYGZgt{}\PYGZgt{} }\PYG{n}{c} \PYG{o}{=} \PYG{n}{cubespa}\PYG{o}{.}\PYG{n}{CubeSPA}\PYG{p}{(}\PYG{n}{filename}\PYG{p}{)}
\end{sphinxVerbatim}

\sphinxAtStartPar
You can load in moment maps (assuming the convention from  \sphinxhref{https://github.com/tonywong94/maskmoment}{maskmoment} ) with the following. If your maskmoment output is
\sphinxcode{\sphinxupquote{path/to/maskmoment.mom0.fits.gz}}, for the moment 0 map (.mom1, .mom2 for the others), these are loaded as follows below. With
moment maps loaded, you can also create a bounding box around “valid” data by calling the \sphinxcode{\sphinxupquote{limits = "auto"}} feature.

\begin{sphinxVerbatim}[commandchars=\\\{\}]
\PYG{g+gp}{\PYGZgt{}\PYGZgt{}\PYGZgt{} }\PYG{n}{filename} \PYG{o}{=} \PYG{l+s+s2}{\PYGZdq{}}\PYG{l+s+s2}{path/to/cube.fits}\PYG{l+s+s2}{\PYGZdq{}}
\PYG{g+gp}{\PYGZgt{}\PYGZgt{}\PYGZgt{} }\PYG{n}{mommaps} \PYG{o}{=} \PYG{l+s+s2}{\PYGZdq{}}\PYG{l+s+s2}{path/to/maskmoment}\PYG{l+s+s2}{\PYGZdq{}}
\PYG{g+gp}{\PYGZgt{}\PYGZgt{}\PYGZgt{} }\PYG{n}{c} \PYG{o}{=} \PYG{n}{cubespa}\PYG{o}{.}\PYG{n}{CubeSPA}\PYG{p}{(}\PYG{n}{cube\PYGZus{}fn}\PYG{p}{,} \PYG{n}{mom\PYGZus{}maps}\PYG{o}{=}\PYG{n}{mommaps}\PYG{p}{,} \PYG{n}{limits}\PYG{o}{=}\PYG{l+s+s2}{\PYGZdq{}}\PYG{l+s+s2}{auto}\PYG{l+s+s2}{\PYGZdq{}}\PYG{p}{)}
\end{sphinxVerbatim}

\sphinxAtStartPar
See getting\_started.rst for more information.


\chapter{Loading Data}
\label{\detokenize{index:loading-data}}
\sphinxstepscope


\section{Getting Started}
\label{\detokenize{getting_started:getting-started}}\label{\detokenize{getting_started::doc}}
\sphinxAtStartPar
The following documentation outlines how to load in data, from an initial cube, to moment maps, to additional data
in both single frame and RGB image form.


\subsection{Initial CubeSPA object}
\label{\detokenize{getting_started:initial-cubespa-object}}
\sphinxAtStartPar
To begin using CubeSPA, create a \sphinxcode{\sphinxupquote{cubespa.CubeSPA}} object with the filename for your cube in the following way:

\begin{sphinxVerbatim}[commandchars=\\\{\}]
\PYG{g+gp}{\PYGZgt{}\PYGZgt{}\PYGZgt{} }\PYG{k+kn}{import} \PYG{n+nn}{cubespa}
\PYG{g+gp}{\PYGZgt{}\PYGZgt{}\PYGZgt{} }\PYG{n}{filename} \PYG{o}{=} \PYG{l+s+s2}{\PYGZdq{}}\PYG{l+s+s2}{path/to/cube.fits}\PYG{l+s+s2}{\PYGZdq{}}
\PYG{g+gp}{\PYGZgt{}\PYGZgt{}\PYGZgt{} }\PYG{n}{c} \PYG{o}{=} \PYG{n}{cubespa}\PYG{o}{.}\PYG{n}{CubeSPA}\PYG{p}{(}\PYG{n}{filename}\PYG{p}{)}
\end{sphinxVerbatim}

\sphinxAtStartPar
You can load in moment maps (assuming the convention from  \sphinxhref{https://github.com/tonywong94/maskmoment}{maskmoment} ) with the following. If your maskmoment output is
\sphinxcode{\sphinxupquote{path/to/maskmoment.mom0.fits.gz}}, for the moment 0 map (.mom1, .mom2 for the others), these are loaded as follows below. With
moment maps loaded, you can also create a bounding box around “valid” data by calling the \sphinxcode{\sphinxupquote{limits = "auto"}} feature.

\begin{sphinxVerbatim}[commandchars=\\\{\}]
\PYG{g+gp}{\PYGZgt{}\PYGZgt{}\PYGZgt{} }\PYG{n}{filename} \PYG{o}{=} \PYG{l+s+s2}{\PYGZdq{}}\PYG{l+s+s2}{path/to/cube.fits}\PYG{l+s+s2}{\PYGZdq{}}
\PYG{g+gp}{\PYGZgt{}\PYGZgt{}\PYGZgt{} }\PYG{n}{mommaps} \PYG{o}{=} \PYG{l+s+s2}{\PYGZdq{}}\PYG{l+s+s2}{path/to/maskmoment}\PYG{l+s+s2}{\PYGZdq{}}
\PYG{g+gp}{\PYGZgt{}\PYGZgt{}\PYGZgt{} }\PYG{n}{c} \PYG{o}{=} \PYG{n}{cubespa}\PYG{o}{.}\PYG{n}{CubeSPA}\PYG{p}{(}\PYG{n}{filename}\PYG{p}{,} \PYG{n}{mom\PYGZus{}maps}\PYG{o}{=}\PYG{n}{mommaps}\PYG{p}{,} \PYG{n}{limits}\PYG{o}{=}\PYG{l+s+s2}{\PYGZdq{}}\PYG{l+s+s2}{auto}\PYG{l+s+s2}{\PYGZdq{}}\PYG{p}{)}
\end{sphinxVerbatim}


\subsection{Additional data}
\label{\detokenize{getting_started:additional-data}}
\sphinxAtStartPar
Additional data that doesn’t require a full cube object can be loaded as a \sphinxcode{\sphinxupquote{cubespa.DataSet()}} object.
Note that all of the data in the cubeSPA object loaded above are also cubespa.DataSet() objects, which stores
the wcs and header information for easier access.

\sphinxAtStartPar
For example, if you had an H\sphinxhyphen{}alpha map of your galaxy, you might load it as follows:

\begin{sphinxVerbatim}[commandchars=\\\{\}]
\PYG{g+gp}{\PYGZgt{}\PYGZgt{}\PYGZgt{} }\PYG{n}{halpha} \PYG{o}{=} \PYG{n}{cubespa}\PYG{o}{.}\PYG{n}{load\PYGZus{}data}\PYG{p}{(}\PYG{n}{halpha\PYGZus{}fn}\PYG{p}{,} \PYG{n}{label}\PYG{o}{=}\PYG{l+s+s2}{\PYGZdq{}}\PYG{l+s+s2}{HALPHA}\PYG{l+s+s2}{\PYGZdq{}}\PYG{p}{)}
\end{sphinxVerbatim}

\sphinxAtStartPar
However, this map might not be aligned with your cube. CubeSPA uses the
\sphinxhref{https://reproject.readthedocs.io/en/stable/}{reproject}
package to properly align maps together. Once you have your \sphinxcode{\sphinxupquote{DataSet}} object, you can align it with

\begin{sphinxVerbatim}[commandchars=\\\{\}]
\PYG{g+gp}{\PYGZgt{}\PYGZgt{}\PYGZgt{} }\PYG{n}{halpha\PYGZus{}interp} \PYG{o}{=} \PYG{n}{cubespa}\PYG{o}{.}\PYG{n}{align\PYGZus{}image}\PYG{p}{(}\PYG{n}{c}\PYG{o}{.}\PYG{n}{mom\PYGZus{}maps}\PYG{o}{.}\PYG{n}{mom0}\PYG{p}{,} \PYG{n}{halpha}\PYG{p}{)}
\end{sphinxVerbatim}

\sphinxAtStartPar
where we are using the moment 0 map to align the images. CubeSPA \sphinxstyleemphasis{should} be able to automatically
align images with a cube directly, where it will do some wcs \sphinxcode{\sphinxupquote{dropaxis}} trickery to try and match things
together.

\sphinxAtStartPar
Lastly, it is good practice to add these datasets to the parent CubeSPA object’s \sphinxcode{\sphinxupquote{additional\_maps}} attribute:

\sphinxAtStartPar
\textgreater{}\textgreater{} c.additional\_maps.extend({[}halpha\_interp, (…){]})


\subsection{RGB images}
\label{\detokenize{getting_started:rgb-images}}
\sphinxAtStartPar
\sphinxcode{\sphinxupquote{matplotlib}} is notoriously tricky for RGB images of astronomical data. CubeSPA has some built\sphinxhyphen{}in features
to improve the experience with displaying RGB data, particularly HST images.

\sphinxAtStartPar
For example, if we load in an hst image and then create an aligned image, we can see what it looks like by
default with the following code:

\begin{sphinxVerbatim}[commandchars=\\\{\}]
\PYG{g+gp}{\PYGZgt{}\PYGZgt{}\PYGZgt{} }\PYG{n}{hst} \PYG{o}{=} \PYG{n}{cubespa}\PYG{o}{.}\PYG{n}{load\PYGZus{}data}\PYG{p}{(}\PYG{n}{hst\PYGZus{}fn}\PYG{p}{,} \PYG{n}{rgb\PYGZus{}index}\PYG{o}{=}\PYG{k+kc}{None}\PYG{p}{,} \PYG{n}{label}\PYG{o}{=}\PYG{l+s+s2}{\PYGZdq{}}\PYG{l+s+s2}{HST}\PYG{l+s+s2}{\PYGZdq{}}\PYG{p}{)}
\PYG{g+gp}{\PYGZgt{}\PYGZgt{}\PYGZgt{} }\PYG{n}{hst\PYGZus{}interp} \PYG{o}{=} \PYG{n}{cubespa}\PYG{o}{.}\PYG{n}{align\PYGZus{}image}\PYG{p}{(}\PYG{n}{c}\PYG{o}{.}\PYG{n}{mom\PYGZus{}maps}\PYG{o}{.}\PYG{n}{mom0}\PYG{p}{,} \PYG{n}{hst}\PYG{p}{)}
\end{sphinxVerbatim}

\begin{sphinxVerbatim}[commandchars=\\\{\}]
\PYG{g+gp}{\PYGZgt{}\PYGZgt{}\PYGZgt{} }\PYG{n}{test} \PYG{o}{=} \PYG{n}{hst\PYGZus{}interp}\PYG{o}{.}\PYG{n}{data}\PYG{o}{.}\PYG{n}{transpose}\PYG{p}{(}\PYG{l+m+mi}{1}\PYG{p}{,}\PYG{l+m+mi}{2}\PYG{p}{,}\PYG{l+m+mi}{0}\PYG{p}{)} \PYG{c+c1}{\PYGZsh{} Transpose the data into the proper rgb pixel format for matplotlib}
\end{sphinxVerbatim}

\begin{sphinxVerbatim}[commandchars=\\\{\}]
\PYG{g+gp}{\PYGZgt{}\PYGZgt{}\PYGZgt{} }\PYG{n}{cubespa}\PYG{o}{.}\PYG{n}{plotting}\PYG{o}{.}\PYG{n}{plot\PYGZus{}rgb}\PYG{p}{(}\PYG{n}{test}\PYG{p}{,} \PYG{n}{lims}\PYG{o}{=}\PYG{n}{np}\PYG{o}{.}\PYG{n}{array}\PYG{p}{(}\PYG{n}{c}\PYG{o}{.}\PYG{n}{limits}\PYG{p}{)}\PYG{p}{,} \PYG{n}{outname}\PYG{o}{=}\PYG{l+s+s2}{\PYGZdq{}}\PYG{l+s+s2}{./rgb\PYGZus{}nonorm.png}\PYG{l+s+s2}{\PYGZdq{}}\PYG{p}{)}
\end{sphinxVerbatim}

\sphinxAtStartPar
This returns the following.

\noindent\sphinxincludegraphics[width=400\sphinxpxdimen]{{rgb_nonorm}.png}

\sphinxAtStartPar
The RGB image can be histogram\sphinxhyphen{}normalized by doing the following:

\begin{sphinxVerbatim}[commandchars=\\\{\}]
\PYG{g+gp}{\PYGZgt{}\PYGZgt{}\PYGZgt{} }\PYG{n}{norm} \PYG{o}{=} \PYG{n}{cubespa}\PYG{o}{.}\PYG{n}{normalized\PYGZus{}rgb\PYGZus{}image}\PYG{p}{(}\PYG{n}{hst\PYGZus{}interp}\PYG{o}{.}\PYG{n}{data}\PYG{p}{,} \PYG{n}{sigma}\PYG{o}{=}\PYG{p}{(}\PYG{l+m+mi}{2}\PYG{p}{,} \PYG{l+m+mi}{5}\PYG{p}{)}\PYG{p}{)}
\PYG{g+gp}{\PYGZgt{}\PYGZgt{}\PYGZgt{} }\PYG{n}{cubespa}\PYG{o}{.}\PYG{n}{plotting}\PYG{o}{.}\PYG{n}{plot\PYGZus{}rgb}\PYG{p}{(}\PYG{n}{norm}\PYG{p}{,} \PYG{n}{lims}\PYG{o}{=}\PYG{n}{np}\PYG{o}{.}\PYG{n}{array}\PYG{p}{(}\PYG{n}{c}\PYG{o}{.}\PYG{n}{limits}\PYG{p}{)}\PYG{p}{)}
\end{sphinxVerbatim}

\sphinxAtStartPar
where \sphinxcode{\sphinxupquote{sigma}} defines the lower and upper bounds to stretch each RGB frame to. This returns a map that
look like:

\noindent\sphinxincludegraphics[width=400\sphinxpxdimen]{{rgb_norm}.png}

\sphinxAtStartPar
If the user wants to look at faint features, simply decrease the upper stretch. This will increase the visibility
of faint features at the cost of saturating the central disk.

\begin{sphinxVerbatim}[commandchars=\\\{\}]
\PYG{g+gp}{\PYGZgt{}\PYGZgt{}\PYGZgt{} }\PYG{n}{norm} \PYG{o}{=} \PYG{n}{cubespa}\PYG{o}{.}\PYG{n}{normalized\PYGZus{}rgb\PYGZus{}image}\PYG{p}{(}\PYG{n}{hst\PYGZus{}interp}\PYG{o}{.}\PYG{n}{data}\PYG{p}{,} \PYG{n}{sigma}\PYG{o}{=}\PYG{p}{(}\PYG{l+m+mi}{2}\PYG{p}{,} \PYG{l+m+mi}{1}\PYG{p}{)}\PYG{p}{)}
\PYG{g+gp}{\PYGZgt{}\PYGZgt{}\PYGZgt{} }\PYG{n}{cubespa}\PYG{o}{.}\PYG{n}{plotting}\PYG{o}{.}\PYG{n}{plot\PYGZus{}rgb}\PYG{p}{(}\PYG{n}{norm}\PYG{p}{,} \PYG{n}{lims}\PYG{o}{=}\PYG{n}{np}\PYG{o}{.}\PYG{n}{array}\PYG{p}{(}\PYG{n}{c}\PYG{o}{.}\PYG{n}{limits}\PYG{p}{)}\PYG{p}{)}
\end{sphinxVerbatim}

\noindent\sphinxincludegraphics[width=400\sphinxpxdimen]{{rgb_faint}.png}


\subsection{Cutouts and Regions}
\label{\detokenize{getting_started:cutouts-and-regions}}
\sphinxAtStartPar
To create specific cutouts (for analysis of certain regions), the user supplies the parent CubeSPA object,
the central location of the cutout, and the size of the cutout (either as an int, for a square, or as a
tuple to establish a rectangular region). The output is a new CubeSPA object where the cube, moment maps,
and any additional maps are trimmed to the location and size of the cutout.

\sphinxAtStartPar
For example, if I was looking at 3 different regions (blob, fallback, and outskirts) for some datacube of a
galaxy, I would create it with the following.

\begin{sphinxVerbatim}[commandchars=\\\{\}]
\PYG{g+gp}{\PYGZgt{}\PYGZgt{}\PYGZgt{} }\PYG{n}{blob} \PYG{o}{=} \PYG{n}{cubespa}\PYG{o}{.}\PYG{n}{gen\PYGZus{}cutout}\PYG{p}{(}\PYG{n}{c}\PYG{p}{,} \PYG{p}{(}\PYG{l+m+mi}{345}\PYG{p}{,} \PYG{l+m+mi}{290}\PYG{p}{)}\PYG{p}{,} \PYG{l+m+mi}{15}\PYG{p}{,} \PYG{n}{show\PYGZus{}bbox}\PYG{o}{=}\PYG{k+kc}{True}\PYG{p}{)}
\PYG{g+gp}{\PYGZgt{}\PYGZgt{}\PYGZgt{} }\PYG{n}{fallback} \PYG{o}{=} \PYG{n}{cubespa}\PYG{o}{.}\PYG{n}{gen\PYGZus{}cutout}\PYG{p}{(}\PYG{n}{c}\PYG{p}{,} \PYG{p}{(}\PYG{l+m+mi}{305}\PYG{p}{,} \PYG{l+m+mi}{310}\PYG{p}{)}\PYG{p}{,} \PYG{p}{(}\PYG{l+m+mi}{20}\PYG{p}{,} \PYG{l+m+mi}{15}\PYG{p}{)}\PYG{p}{,} \PYG{n}{show\PYGZus{}bbox}\PYG{o}{=}\PYG{k+kc}{True}\PYG{p}{)}
\PYG{g+gp}{\PYGZgt{}\PYGZgt{}\PYGZgt{} }\PYG{n}{outskirts} \PYG{o}{=} \PYG{n}{cubespa}\PYG{o}{.}\PYG{n}{gen\PYGZus{}cutout}\PYG{p}{(}\PYG{n}{c}\PYG{p}{,} \PYG{p}{(}\PYG{l+m+mi}{270}\PYG{p}{,} \PYG{l+m+mi}{403}\PYG{p}{)}\PYG{p}{,} \PYG{p}{(}\PYG{l+m+mi}{15}\PYG{p}{,} \PYG{l+m+mi}{35}\PYG{p}{)}\PYG{p}{,} \PYG{n}{show\PYGZus{}bbox}\PYG{o}{=}\PYG{k+kc}{True}\PYG{p}{)}
\end{sphinxVerbatim}

\sphinxAtStartPar
The additional parameter \sphinxcode{\sphinxupquote{show\_bbox}} will generate a plot to show you where the cutout falls on the parent
image. This is helpful for more closely aligning the cutouts.


\chapter{2D Analysis}
\label{\detokenize{index:d-analysis}}
\sphinxstepscope


\section{Generating Moment Maps}
\label{\detokenize{momentmaps:generating-moment-maps}}\label{\detokenize{momentmaps::doc}}
\sphinxAtStartPar
Generating moment maps is trivially easy using CubeSPA.

\sphinxAtStartPar
To make a display of the moment maps, simply do the following:

\begin{sphinxVerbatim}[commandchars=\\\{\}]
\PYG{g+gp}{\PYGZgt{}\PYGZgt{}\PYGZgt{} }\PYG{k+kn}{import} \PYG{n+nn}{cubespa}
\PYG{g+gp}{\PYGZgt{}\PYGZgt{}\PYGZgt{} }\PYG{n}{c} \PYG{o}{=} \PYG{n}{cubespa}\PYG{o}{.}\PYG{n}{CubeSPA}\PYG{p}{(}\PYG{n}{cube\PYGZus{}filename}\PYG{p}{,} \PYG{n}{mom\PYGZus{}maps}\PYG{o}{=}\PYG{n}{mom\PYGZus{}maps\PYGZus{}filename}\PYG{p}{)}
\PYG{g+gp}{\PYGZgt{}\PYGZgt{}\PYGZgt{} }\PYG{n}{c}\PYG{o}{.}\PYG{n}{plot\PYGZus{}moment\PYGZus{}maps}\PYG{p}{(}\PYG{p}{)}
\end{sphinxVerbatim}

\sphinxAtStartPar
If you have made some cutouts of your object, you can do the same, but it is recommended that you
set \sphinxcode{\sphinxupquote{use\_limits}} to \sphinxcode{\sphinxupquote{False}} as the limits are related directly to the parent cube object.

\begin{sphinxVerbatim}[commandchars=\\\{\}]
\PYG{g+gp}{\PYGZgt{}\PYGZgt{}\PYGZgt{} }\PYG{n}{blob}\PYG{o}{.}\PYG{n}{plot\PYGZus{}moment\PYGZus{}maps}\PYG{p}{(}\PYG{n}{use\PYGZus{}limits}\PYG{o}{=}\PYG{k+kc}{False}\PYG{p}{)}
\end{sphinxVerbatim}

\noindent\sphinxincludegraphics[width=800\sphinxpxdimen]{{blob}.png}


\subsection{Reference/API}
\label{\detokenize{momentmaps:module-cubespa.plotting.mommap_plots}}\label{\detokenize{momentmaps:reference-api}}\index{module@\spxentry{module}!cubespa.plotting.mommap\_plots@\spxentry{cubespa.plotting.mommap\_plots}}\index{cubespa.plotting.mommap\_plots@\spxentry{cubespa.plotting.mommap\_plots}!module@\spxentry{module}}\index{moment\_map\_plot() (in module cubespa.plotting.mommap\_plots)@\spxentry{moment\_map\_plot()}\spxextra{in module cubespa.plotting.mommap\_plots}}

\begin{fulllineitems}
\phantomsection\label{\detokenize{momentmaps:cubespa.plotting.mommap_plots.moment_map_plot}}
\pysigstartsignatures
\pysiglinewithargsret{\sphinxcode{\sphinxupquote{cubespa.plotting.mommap\_plots.}}\sphinxbfcode{\sphinxupquote{moment\_map\_plot}}}{\emph{\DUrole{n}{cubespa\_obj}}, \emph{\DUrole{n}{filename}\DUrole{o}{=}\DUrole{default_value}{None}}, \emph{\DUrole{n}{use\_limits}\DUrole{o}{=}\DUrole{default_value}{True}}, \emph{\DUrole{o}{**}\DUrole{n}{kwargs}}}{}
\pysigstopsignatures
\sphinxAtStartPar
Generate moment map plots.
\begin{quote}\begin{description}
\sphinxlineitem{Parameters}\begin{itemize}
\item {} 
\sphinxAtStartPar
\sphinxstyleliteralstrong{\sphinxupquote{cubespa\_obj}} ({\hyperref[\detokenize{modules:cubespa.cubespa.CubeSPA}]{\sphinxcrossref{\sphinxstyleliteralemphasis{\sphinxupquote{cubespa.CubeSPA}}}}}) \textendash{} The input CubeSPA object, with valid moment maps loaded.

\item {} 
\sphinxAtStartPar
\sphinxstyleliteralstrong{\sphinxupquote{filename}} (\sphinxstyleliteralemphasis{\sphinxupquote{str}}\sphinxstyleliteralemphasis{\sphinxupquote{, }}\sphinxstyleliteralemphasis{\sphinxupquote{optional}}) \textendash{} Output filename, in which the plot is just shown instead of saved. Defaults to None.

\item {} 
\sphinxAtStartPar
\sphinxstyleliteralstrong{\sphinxupquote{use\_limits}} (\sphinxstyleliteralemphasis{\sphinxupquote{bool}}\sphinxstyleliteralemphasis{\sphinxupquote{, }}\sphinxstyleliteralemphasis{\sphinxupquote{optional}}) \textendash{} 
\sphinxAtStartPar
Whether or not to use limits from the CubeSPA object.

\sphinxAtStartPar
It is a good idea to set to False for cutout objects, as their limits will be relative to the initial CubeSPA object,
and their desired limits will be the entire array. Defaults to True.


\end{itemize}

\end{description}\end{quote}

\end{fulllineitems}


\sphinxstepscope


\section{Making Overlays}
\label{\detokenize{overlays:making-overlays}}\label{\detokenize{overlays::doc}}
\sphinxAtStartPar
Overlays are useful features to display isophotal contours of one image onto another.

\sphinxAtStartPar
Once you have aligned datasets (see getting\_started), you can create an overlay by using:

\begin{sphinxVerbatim}[commandchars=\\\{\}]
\PYG{g+gp}{\PYGZgt{}\PYGZgt{}\PYGZgt{} }\PYG{n}{cubespa}\PYG{o}{.}\PYG{n}{plotting}\PYG{o}{.}\PYG{n}{overlay\PYGZus{}plot}\PYG{p}{(}\PYG{n}{c}\PYG{o}{.}\PYG{n}{mom\PYGZus{}maps}\PYG{o}{.}\PYG{n}{mom0}\PYG{p}{,} \PYG{n}{c}\PYG{o}{.}\PYG{n}{additional\PYGZus{}maps}\PYG{p}{[}\PYG{l+m+mi}{0}\PYG{p}{]}\PYG{p}{,}
\PYG{g+gp}{... }                              \PYG{n}{lims}\PYG{o}{=}\PYG{n}{c}\PYG{o}{.}\PYG{n}{limits}\PYG{p}{,} \PYG{n}{levels}\PYG{o}{=}\PYG{p}{[}\PYG{l+m+mi}{10}\PYG{p}{,} \PYG{l+m+mi}{20}\PYG{p}{,} \PYG{l+m+mi}{100}\PYG{p}{,} \PYG{l+m+mi}{150}\PYG{p}{]}\PYG{p}{)}
\end{sphinxVerbatim}

\sphinxAtStartPar
where in this instance, we are showing the moment 0 CO distribution with H\sphinxhyphen{}alpha contours overlaid in black.
The output plot looks like this:

\noindent\sphinxincludegraphics[width=600\sphinxpxdimen]{{test_overlay}.png}

\sphinxAtStartPar
For an RGB image, use the following instead:

\begin{sphinxVerbatim}[commandchars=\\\{\}]
\PYG{g+gp}{\PYGZgt{}\PYGZgt{}\PYGZgt{} }\PYG{n}{norm} \PYG{o}{=} \PYG{n}{cubespa}\PYG{o}{.}\PYG{n}{normalized\PYGZus{}rgb\PYGZus{}image}\PYG{p}{(}\PYG{n}{hst\PYGZus{}interp}\PYG{o}{.}\PYG{n}{data}\PYG{p}{,} \PYG{n}{sigma}\PYG{o}{=}\PYG{p}{(}\PYG{l+m+mi}{2}\PYG{p}{,} \PYG{l+m+mi}{1}\PYG{p}{)}\PYG{p}{)}
\PYG{g+gp}{\PYGZgt{}\PYGZgt{}\PYGZgt{} }\PYG{n}{cubespa}\PYG{o}{.}\PYG{n}{plotting}\PYG{o}{.}\PYG{n}{rgb\PYGZus{}overlay}\PYG{p}{(}\PYG{n}{norm}\PYG{p}{,} \PYG{n}{c}\PYG{o}{.}\PYG{n}{mom\PYGZus{}maps}\PYG{o}{.}\PYG{n}{mom0}\PYG{p}{,}
\PYG{g+gp}{... }                             \PYG{n}{levels}\PYG{o}{=}\PYG{p}{[}\PYG{l+m+mf}{0.02}\PYG{p}{,} \PYG{l+m+mf}{0.05}\PYG{p}{,} \PYG{l+m+mf}{0.1}\PYG{p}{,} \PYG{l+m+mi}{1}\PYG{p}{]}\PYG{p}{,} \PYG{n}{lims}\PYG{o}{=}\PYG{n}{blob}\PYG{o}{.}\PYG{n}{limits}\PYG{p}{,} \PYG{n}{colors}\PYG{o}{=}\PYG{l+s+s2}{\PYGZdq{}}\PYG{l+s+s2}{white}\PYG{l+s+s2}{\PYGZdq{}}\PYG{p}{)}
\end{sphinxVerbatim}

\noindent\sphinxincludegraphics[width=600\sphinxpxdimen]{{rgb_overlay}.png}

\sphinxAtStartPar
which plots CO contours on top of the HST RGB image, which for this region reveals
a small compact stellar feature at the head of the  “blob”.


\subsection{PSF Overlays for Sidelobe Checking}
\label{\detokenize{overlays:psf-overlays-for-sidelobe-checking}}
\sphinxAtStartPar
If your cube/cleaning has a PSF that you have imported into cubespa with the \sphinxcode{\sphinxupquote{psf=psf\_fn}} feature, you can overplot it using the following
code:

\begin{sphinxVerbatim}[commandchars=\\\{\}]
\PYG{g+gp}{\PYGZgt{}\PYGZgt{}\PYGZgt{} }\PYG{n}{c} \PYG{o}{=} \PYG{n}{cubespa}\PYG{o}{.}\PYG{n}{CubeSPA}\PYG{p}{(}\PYG{n}{cube}\PYG{o}{=}\PYG{n}{cube\PYGZus{}fn}\PYG{p}{,} \PYG{n}{psf}\PYG{o}{=}\PYG{n}{psf\PYGZus{}fn}\PYG{p}{,} \PYG{n}{limits}\PYG{o}{=}\PYG{l+s+s2}{\PYGZdq{}}\PYG{l+s+s2}{auto}\PYG{l+s+s2}{\PYGZdq{}}\PYG{p}{)}
\PYG{g+gp}{\PYGZgt{}\PYGZgt{}\PYGZgt{} }\PYG{n}{cubespa}\PYG{o}{.}\PYG{n}{plotting}\PYG{o}{.}\PYG{n}{plot\PYGZus{}psf\PYGZus{}overlay}\PYG{p}{(}\PYG{n}{c}\PYG{p}{,} \PYG{n}{y0}\PYG{o}{=}\PYG{l+m+mi}{102}\PYG{p}{,} \PYG{n}{x0}\PYG{o}{=}\PYG{l+m+mi}{25}\PYG{p}{)}
\end{sphinxVerbatim}

\noindent\sphinxincludegraphics[width=600\sphinxpxdimen]{{psf_overlay}.png}

\sphinxAtStartPar
Note that in this case, the psf has been shifted down by 102 pixels, and to the right by 25 pixels. This overlay is especially helpful when hunting
for sidelobe features in a dataset, which can be revealed by looking at contours in the PSF.


\chapter{Cube Analysis}
\label{\detokenize{index:cube-analysis}}
\sphinxstepscope


\section{Comparing Two Datacubes}
\label{\detokenize{cube_comparison:comparing-two-datacubes}}\label{\detokenize{cube_comparison::doc}}
\sphinxAtStartPar
Comparing two different datacubes is an important step for any analysis, whether it is cubes from a different
cleaning algorithm (the PHANGs pipeline vs a manual CASA cleaning, for example),
or different momemt mask parameters,
or cubes cleaned with different parameters, seeing how these cubes compare directly with one another is
extremely important.

\sphinxAtStartPar
To get started comparing two cubes, one uses the \sphinxcode{\sphinxupquote{CubeComparison}} object.

\sphinxstepscope


\section{Spectral Analysis}
\label{\detokenize{spectra:spectral-analysis}}\label{\detokenize{spectra::doc}}
\sphinxAtStartPar
To generate spectra, the user specifies the location of the aperture (elliptical by default), and the shape.
Continuing with our \sphinxcode{\sphinxupquote{blob}} example, we generate spectra like so:

\begin{sphinxVerbatim}[commandchars=\\\{\}]
\PYG{g+gp}{\PYGZgt{}\PYGZgt{}\PYGZgt{} }\PYG{n}{blob}\PYG{o}{.}\PYG{n}{create\PYGZus{}spectra}\PYG{p}{(}\PYG{p}{(}\PYG{l+m+mi}{15}\PYG{p}{,} \PYG{l+m+mi}{13}\PYG{p}{)}\PYG{p}{,} \PYG{p}{(}\PYG{l+m+mi}{6}\PYG{p}{,} \PYG{l+m+mi}{6}\PYG{p}{)}\PYG{p}{,} \PYG{n}{plot}\PYG{o}{=}\PYG{k+kc}{True}\PYG{p}{)}
\end{sphinxVerbatim}

\sphinxAtStartPar
This creates the following plot, showing the location of the spectral features in either velocity or channel
space. CubeSPA uses sigma clipping to determine an RMS of the spectra, and then plots the 1,2 and 3\sphinxhyphen{}sigma levels
as a shaded grey region.

\noindent\sphinxincludegraphics[width=800\sphinxpxdimen]{{blob_spectra}.png}


\subsection{SNR Analysis}
\label{\detokenize{spectra:snr-analysis}}
\sphinxAtStartPar
To manually check the SNR of a given region (say, between channels 20 and 40 of the array \sphinxcode{\sphinxupquote{spec}}),
you can with the following method:

\begin{sphinxVerbatim}[commandchars=\\\{\}]
\PYG{g+gp}{\PYGZgt{}\PYGZgt{}\PYGZgt{} }\PYG{n}{chan\PYGZus{}min}\PYG{p}{,} \PYG{n}{chan\PYGZus{}max} \PYG{o}{=} \PYG{l+m+mi}{20}\PYG{p}{,} \PYG{l+m+mi}{40}
\PYG{g+gp}{\PYGZgt{}\PYGZgt{}\PYGZgt{} }\PYG{n}{snr} \PYG{o}{=} \PYG{n}{cubespa}\PYG{o}{.}\PYG{n}{calc\PYGZus{}snr}\PYG{p}{(}\PYG{n}{spec}\PYG{p}{,} \PYG{n}{chan\PYGZus{}min}\PYG{p}{,} \PYG{n}{chan\PYGZus{}max}\PYG{p}{)}
\end{sphinxVerbatim}

\sphinxAtStartPar
The signal to noise ratio between the two channels is given by:
\begin{equation*}
\begin{split}SNR = \frac{\Sigma F}{RMS_{spec} * \sqrt{n_{chan}}}\end{split}
\end{equation*}\phantomsection\label{\detokenize{spectra:module-cubespa.plotting.spectra_plots}}\index{module@\spxentry{module}!cubespa.plotting.spectra\_plots@\spxentry{cubespa.plotting.spectra\_plots}}\index{cubespa.plotting.spectra\_plots@\spxentry{cubespa.plotting.spectra\_plots}!module@\spxentry{module}}\index{multispec\_plot() (in module cubespa.plotting.spectra\_plots)@\spxentry{multispec\_plot()}\spxextra{in module cubespa.plotting.spectra\_plots}}

\begin{fulllineitems}
\phantomsection\label{\detokenize{spectra:cubespa.plotting.spectra_plots.multispec_plot}}
\pysigstartsignatures
\pysiglinewithargsret{\sphinxcode{\sphinxupquote{cubespa.plotting.spectra\_plots.}}\sphinxbfcode{\sphinxupquote{multispec\_plot}}}{\emph{\DUrole{n}{cubespa\_obj}}, \emph{\DUrole{n}{aper\_list}}, \emph{\DUrole{n}{spec\_list}}, \emph{\DUrole{o}{**}\DUrole{n}{kwargs}}}{}
\pysigstopsignatures
\end{fulllineitems}

\index{plot\_spectra() (in module cubespa.plotting.spectra\_plots)@\spxentry{plot\_spectra()}\spxextra{in module cubespa.plotting.spectra\_plots}}

\begin{fulllineitems}
\phantomsection\label{\detokenize{spectra:cubespa.plotting.spectra_plots.plot_spectra}}
\pysigstartsignatures
\pysiglinewithargsret{\sphinxcode{\sphinxupquote{cubespa.plotting.spectra\_plots.}}\sphinxbfcode{\sphinxupquote{plot\_spectra}}}{\emph{\DUrole{n}{data}}, \emph{\DUrole{n}{aper}}}{}
\pysigstopsignatures
\end{fulllineitems}

\index{spectra\_comparison() (in module cubespa.plotting.spectra\_plots)@\spxentry{spectra\_comparison()}\spxextra{in module cubespa.plotting.spectra\_plots}}

\begin{fulllineitems}
\phantomsection\label{\detokenize{spectra:cubespa.plotting.spectra_plots.spectra_comparison}}
\pysigstartsignatures
\pysiglinewithargsret{\sphinxcode{\sphinxupquote{cubespa.plotting.spectra\_plots.}}\sphinxbfcode{\sphinxupquote{spectra\_comparison}}}{\emph{\DUrole{n}{cubecomp}}, \emph{\DUrole{n}{a1}}, \emph{\DUrole{n}{a2}}, \emph{\DUrole{n}{s1}}, \emph{\DUrole{n}{s2}}, \emph{\DUrole{n}{chan\_ranges}\DUrole{o}{=}\DUrole{default_value}{None}}, \emph{\DUrole{o}{**}\DUrole{n}{kwargs}}}{}
\pysigstopsignatures
\end{fulllineitems}

\index{spectra\_plot() (in module cubespa.plotting.spectra\_plots)@\spxentry{spectra\_plot()}\spxextra{in module cubespa.plotting.spectra\_plots}}

\begin{fulllineitems}
\phantomsection\label{\detokenize{spectra:cubespa.plotting.spectra_plots.spectra_plot}}
\pysigstartsignatures
\pysiglinewithargsret{\sphinxcode{\sphinxupquote{cubespa.plotting.spectra\_plots.}}\sphinxbfcode{\sphinxupquote{spectra\_plot}}}{\emph{\DUrole{n}{cubespa\_obj}}, \emph{\DUrole{n}{aper}}, \emph{\DUrole{n}{spectrum}}}{}
\pysigstopsignatures\begin{description}
\sphinxlineitem{Create a plot showing both the image with overlaid spectra, as well as the spectrum with}
\sphinxAtStartPar
RMS levels shown.

\end{description}
\begin{quote}\begin{description}
\sphinxlineitem{Parameters}\begin{itemize}
\item {} 
\sphinxAtStartPar
\sphinxstyleliteralstrong{\sphinxupquote{cubespa\_obj}} ({\hyperref[\detokenize{modules:cubespa.cubespa.CubeSPA}]{\sphinxcrossref{\sphinxstyleliteralemphasis{\sphinxupquote{cubespa.CubeSPA}}}}}) \textendash{} CubeSPA object.

\item {} 
\sphinxAtStartPar
\sphinxstyleliteralstrong{\sphinxupquote{aper}} (\sphinxstyleliteralemphasis{\sphinxupquote{photutils.aperture}}) \textendash{} Input aperture, generated using cubespa.spectra

\item {} 
\sphinxAtStartPar
\sphinxstyleliteralstrong{\sphinxupquote{spectrum}} (\sphinxstyleliteralemphasis{\sphinxupquote{\_type\_}}) \textendash{} \_description\_

\end{itemize}

\end{description}\end{quote}

\end{fulllineitems}

\phantomsection\label{\detokenize{spectra:module-cubespa.spectra}}\index{module@\spxentry{module}!cubespa.spectra@\spxentry{cubespa.spectra}}\index{cubespa.spectra@\spxentry{cubespa.spectra}!module@\spxentry{module}}\index{align\_apertures() (in module cubespa.spectra)@\spxentry{align\_apertures()}\spxextra{in module cubespa.spectra}}

\begin{fulllineitems}
\phantomsection\label{\detokenize{spectra:cubespa.spectra.align_apertures}}
\pysigstartsignatures
\pysiglinewithargsret{\sphinxcode{\sphinxupquote{cubespa.spectra.}}\sphinxbfcode{\sphinxupquote{align\_apertures}}}{\emph{\DUrole{n}{aper\_list}}, \emph{\DUrole{n}{wcs1}}, \emph{\DUrole{n}{wcs2}}}{}
\pysigstopsignatures
\sphinxAtStartPar
Align and resize a set of apertures from wcs1 to wcs2
\begin{quote}\begin{description}
\sphinxlineitem{Parameters}\begin{itemize}
\item {} 
\sphinxAtStartPar
\sphinxstyleliteralstrong{\sphinxupquote{aper\_list}} (\sphinxstyleliteralemphasis{\sphinxupquote{list}}) \textendash{} List of apertures in the following format:
{[}(p1, p2), (s1, s2){]} where the first tuple is the (x,y) position and the second tuple is the (x,y) height.

\item {} 
\sphinxAtStartPar
\sphinxstyleliteralstrong{\sphinxupquote{wcs1}} (\sphinxstyleliteralemphasis{\sphinxupquote{astropy.wcs.WCS}}) \textendash{} WCS that aper\_list apertures are already aligned to

\item {} 
\sphinxAtStartPar
\sphinxstyleliteralstrong{\sphinxupquote{wcs2}} (\sphinxstyleliteralemphasis{\sphinxupquote{astropy.wcs.WCS}}) \textendash{} WCS to align apertures to

\end{itemize}

\sphinxlineitem{Returns}
\sphinxAtStartPar
\_description\_

\sphinxlineitem{Return type}
\sphinxAtStartPar
\_type\_

\end{description}\end{quote}

\end{fulllineitems}

\index{analyze\_spectra() (in module cubespa.spectra)@\spxentry{analyze\_spectra()}\spxextra{in module cubespa.spectra}}

\begin{fulllineitems}
\phantomsection\label{\detokenize{spectra:cubespa.spectra.analyze_spectra}}
\pysigstartsignatures
\pysiglinewithargsret{\sphinxcode{\sphinxupquote{cubespa.spectra.}}\sphinxbfcode{\sphinxupquote{analyze\_spectra}}}{\emph{\DUrole{n}{spec}}, \emph{\DUrole{n}{sigma}\DUrole{o}{=}\DUrole{default_value}{2}}, \emph{\DUrole{n}{cmin}\DUrole{o}{=}\DUrole{default_value}{None}}, \emph{\DUrole{n}{cmax}\DUrole{o}{=}\DUrole{default_value}{None}}}{}
\pysigstopsignatures
\end{fulllineitems}

\index{calc\_snr() (in module cubespa.spectra)@\spxentry{calc\_snr()}\spxextra{in module cubespa.spectra}}

\begin{fulllineitems}
\phantomsection\label{\detokenize{spectra:cubespa.spectra.calc_snr}}
\pysigstartsignatures
\pysiglinewithargsret{\sphinxcode{\sphinxupquote{cubespa.spectra.}}\sphinxbfcode{\sphinxupquote{calc\_snr}}}{\emph{\DUrole{n}{spec}}, \emph{\DUrole{n}{chan\_min}}, \emph{\DUrole{n}{chan\_max}}}{}
\pysigstopsignatures
\end{fulllineitems}

\index{create\_aperture() (in module cubespa.spectra)@\spxentry{create\_aperture()}\spxextra{in module cubespa.spectra}}

\begin{fulllineitems}
\phantomsection\label{\detokenize{spectra:cubespa.spectra.create_aperture}}
\pysigstartsignatures
\pysiglinewithargsret{\sphinxcode{\sphinxupquote{cubespa.spectra.}}\sphinxbfcode{\sphinxupquote{create\_aperture}}}{\emph{\DUrole{n}{cubespa\_obj}}, \emph{\DUrole{n}{position}}, \emph{\DUrole{n}{shape}}, \emph{\DUrole{n}{aper\_type}\DUrole{o}{=}\DUrole{default_value}{\textquotesingle{}elliptical\textquotesingle{}}}, \emph{\DUrole{n}{plot}\DUrole{o}{=}\DUrole{default_value}{False}}}{}
\pysigstopsignatures
\sphinxAtStartPar
Generate photutils aperture of desired type, position, and shape.
\begin{quote}\begin{description}
\sphinxlineitem{Returns}
\sphinxAtStartPar
Photutils aperture

\sphinxlineitem{Return type}
\sphinxAtStartPar
photutils.aperture

\end{description}\end{quote}

\end{fulllineitems}

\index{get\_spectra() (in module cubespa.spectra)@\spxentry{get\_spectra()}\spxextra{in module cubespa.spectra}}

\begin{fulllineitems}
\phantomsection\label{\detokenize{spectra:cubespa.spectra.get_spectra}}
\pysigstartsignatures
\pysiglinewithargsret{\sphinxcode{\sphinxupquote{cubespa.spectra.}}\sphinxbfcode{\sphinxupquote{get\_spectra}}}{\emph{\DUrole{n}{cube}}, \emph{\DUrole{n}{aper}}}{}
\pysigstopsignatures
\sphinxAtStartPar
Get the spectra through a datacube at the position and size of a given aperture.
\begin{quote}\begin{description}
\sphinxlineitem{Parameters}\begin{itemize}
\item {} 
\sphinxAtStartPar
\sphinxstyleliteralstrong{\sphinxupquote{cube}} (\sphinxstyleliteralemphasis{\sphinxupquote{ndarray}}) \textendash{} \_description\_

\item {} 
\sphinxAtStartPar
\sphinxstyleliteralstrong{\sphinxupquote{aper}} (\sphinxstyleliteralemphasis{\sphinxupquote{photutils aperture}}) \textendash{} Elliptical or circular aperture/annulus.

\end{itemize}

\sphinxlineitem{Returns}
\sphinxAtStartPar
\_description\_

\sphinxlineitem{Return type}
\sphinxAtStartPar
\_type\_

\end{description}\end{quote}

\end{fulllineitems}

\index{multi\_spec() (in module cubespa.spectra)@\spxentry{multi\_spec()}\spxextra{in module cubespa.spectra}}

\begin{fulllineitems}
\phantomsection\label{\detokenize{spectra:cubespa.spectra.multi_spec}}
\pysigstartsignatures
\pysiglinewithargsret{\sphinxcode{\sphinxupquote{cubespa.spectra.}}\sphinxbfcode{\sphinxupquote{multi\_spec}}}{\emph{\DUrole{n}{cubespa\_obj}}, \emph{\DUrole{n}{spec\_info}}}{}
\pysigstopsignatures
\sphinxAtStartPar
Generate a list of apertures and spectra for better diagnostics.
\begin{quote}\begin{description}
\sphinxlineitem{Parameters}\begin{itemize}
\item {} 
\sphinxAtStartPar
\sphinxstyleliteralstrong{\sphinxupquote{cubespa\_obj}} ({\hyperref[\detokenize{modules:cubespa.cubespa.CubeSPA}]{\sphinxcrossref{\sphinxstyleliteralemphasis{\sphinxupquote{cubespa.CubeSPA}}}}}) \textendash{} Input CubeSPA object to pull spectra from.

\item {} 
\sphinxAtStartPar
\sphinxstyleliteralstrong{\sphinxupquote{spec\_info}} (\sphinxstyleliteralemphasis{\sphinxupquote{\_type\_}}) \textendash{} A list of spectra position and sizes, for example, to get two spectra, you would

\item {} 
\sphinxAtStartPar
\sphinxstyleliteralstrong{\sphinxupquote{have}} \textendash{} \begin{description}
\sphinxlineitem{spec\_info = {[}{[}(p1, p2), (s1, s2){]},}
\sphinxAtStartPar
{[}(p3, p4), (s3, s4){]}{]}

\end{description}


\end{itemize}

\sphinxlineitem{Returns}
\sphinxAtStartPar
A list of apertures and a list of spectra.

\sphinxlineitem{Return type}
\sphinxAtStartPar
tuple(array)

\end{description}\end{quote}

\end{fulllineitems}

\index{spectra\_comparison() (in module cubespa.spectra)@\spxentry{spectra\_comparison()}\spxextra{in module cubespa.spectra}}

\begin{fulllineitems}
\phantomsection\label{\detokenize{spectra:cubespa.spectra.spectra_comparison}}
\pysigstartsignatures
\pysiglinewithargsret{\sphinxcode{\sphinxupquote{cubespa.spectra.}}\sphinxbfcode{\sphinxupquote{spectra\_comparison}}}{\emph{\DUrole{n}{cubecomp}}, \emph{\DUrole{n}{aper\_list}}, \emph{\DUrole{n}{outname}\DUrole{o}{=}\DUrole{default_value}{None}}, \emph{\DUrole{n}{plot\_ticks}\DUrole{o}{=}\DUrole{default_value}{True}}, \emph{\DUrole{n}{chan\_ranges}\DUrole{o}{=}\DUrole{default_value}{None}}}{}
\pysigstopsignatures
\end{fulllineitems}


\sphinxstepscope


\section{Position\sphinxhyphen{}Velocity diagrams}
\label{\detokenize{pvds:position-velocity-diagrams}}\label{\detokenize{pvds::doc}}
\sphinxAtStartPar
Position\sphinxhyphen{}velocity diagrams (PVDs) plot the strength of velocity features along some specified path, most often
along the major and minor axes. A disk with only circular motions will show the rotation curve along the
major axis, and should be flat along the minor axis. Therefore, PVDs are useful tools to find non\sphinxhyphen{}circular
velocity features without explicit modeling of components.

\sphinxAtStartPar
To generate a PVD, do the following:

\begin{sphinxVerbatim}[commandchars=\\\{\}]
\PYG{g+gp}{\PYGZgt{}\PYGZgt{}\PYGZgt{} }\PYG{n}{pvd} \PYG{o}{=} \PYG{n}{cubespa}\PYG{o}{.}\PYG{n}{gen\PYGZus{}pvd}\PYG{p}{(}\PYG{n}{blob}\PYG{p}{,} \PYG{p}{(}\PYG{l+m+mi}{15}\PYG{p}{,} \PYG{l+m+mi}{12}\PYG{p}{)}\PYG{p}{,} \PYG{l+m+mi}{25}\PYG{p}{,} \PYG{l+m+mi}{0}\PYG{p}{,} \PYG{n}{width}\PYG{o}{=}\PYG{l+m+mi}{8}\PYG{p}{)}
\PYG{g+gp}{\PYGZgt{}\PYGZgt{}\PYGZgt{} }\PYG{n}{cubespa}\PYG{o}{.}\PYG{n}{plotting}\PYG{o}{.}\PYG{n}{pvd\PYGZus{}plot}\PYG{p}{(}\PYG{n}{blob}\PYG{p}{,} \PYG{n}{pvd}\PYG{p}{,} \PYG{n}{vmin}\PYG{o}{=}\PYG{o}{\PYGZhy{}}\PYG{l+m+mi}{125}\PYG{p}{)}
\end{sphinxVerbatim}

\sphinxAtStartPar
Wich will spit out both the map of the region with the overlaid PVD path (with proper width displayed), as
well as the PVD itself. In the example, the region of the PVD cutting through the “blob” shows emission in
blue\sphinxhyphen{}shifted velocity channels, consistent with what is seen in the moment 1 mean velocity map.

\noindent\sphinxincludegraphics[width=600\sphinxpxdimen]{{example_pvd}.png}


\chapter{Coming soon}
\label{\detokenize{index:coming-soon}}
\sphinxstepscope


\section{Generating / Fitting Velocity Models}
\label{\detokenize{modeling:generating-fitting-velocity-models}}\label{\detokenize{modeling::doc}}

\chapter{Indices and tables}
\label{\detokenize{index:indices-and-tables}}
\sphinxstepscope


\section{Module Documentation}
\label{\detokenize{modules:module-documentation}}\label{\detokenize{modules::doc}}

\subsection{Utilities}
\label{\detokenize{modules:module-cubespa.cubespa}}\label{\detokenize{modules:utilities}}\index{module@\spxentry{module}!cubespa.cubespa@\spxentry{cubespa.cubespa}}\index{cubespa.cubespa@\spxentry{cubespa.cubespa}!module@\spxentry{module}}\index{CubeComparison (class in cubespa.cubespa)@\spxentry{CubeComparison}\spxextra{class in cubespa.cubespa}}

\begin{fulllineitems}
\phantomsection\label{\detokenize{modules:cubespa.cubespa.CubeComparison}}
\pysigstartsignatures
\pysiglinewithargsret{\sphinxbfcode{\sphinxupquote{class\DUrole{w}{  }}}\sphinxcode{\sphinxupquote{cubespa.cubespa.}}\sphinxbfcode{\sphinxupquote{CubeComparison}}}{\emph{\DUrole{n}{cube1}\DUrole{p}{:}\DUrole{w}{  }\DUrole{n}{{\hyperref[\detokenize{modules:cubespa.cubespa.CubeSPA}]{\sphinxcrossref{CubeSPA}}}}}, \emph{\DUrole{n}{cube2}\DUrole{p}{:}\DUrole{w}{  }\DUrole{n}{{\hyperref[\detokenize{modules:cubespa.cubespa.CubeSPA}]{\sphinxcrossref{CubeSPA}}}}}}{}
\pysigstopsignatures
\sphinxAtStartPar
Bases: \sphinxcode{\sphinxupquote{object}}
\index{create\_spectra() (cubespa.cubespa.CubeComparison method)@\spxentry{create\_spectra()}\spxextra{cubespa.cubespa.CubeComparison method}}

\begin{fulllineitems}
\phantomsection\label{\detokenize{modules:cubespa.cubespa.CubeComparison.create_spectra}}
\pysigstartsignatures
\pysiglinewithargsret{\sphinxbfcode{\sphinxupquote{create\_spectra}}}{\emph{\DUrole{n}{position}}, \emph{\DUrole{n}{size}}, \emph{\DUrole{n}{return\_products}\DUrole{o}{=}\DUrole{default_value}{False}}, \emph{\DUrole{n}{plot}\DUrole{o}{=}\DUrole{default_value}{False}}}{}
\pysigstopsignatures
\end{fulllineitems}

\index{readout() (cubespa.cubespa.CubeComparison method)@\spxentry{readout()}\spxextra{cubespa.cubespa.CubeComparison method}}

\begin{fulllineitems}
\phantomsection\label{\detokenize{modules:cubespa.cubespa.CubeComparison.readout}}
\pysigstartsignatures
\pysiglinewithargsret{\sphinxbfcode{\sphinxupquote{readout}}}{}{}
\pysigstopsignatures
\end{fulllineitems}


\end{fulllineitems}

\index{CubeSPA (class in cubespa.cubespa)@\spxentry{CubeSPA}\spxextra{class in cubespa.cubespa}}

\begin{fulllineitems}
\phantomsection\label{\detokenize{modules:cubespa.cubespa.CubeSPA}}
\pysigstartsignatures
\pysiglinewithargsret{\sphinxbfcode{\sphinxupquote{class\DUrole{w}{  }}}\sphinxcode{\sphinxupquote{cubespa.cubespa.}}\sphinxbfcode{\sphinxupquote{CubeSPA}}}{\emph{\DUrole{n}{cube}}, \emph{\DUrole{n}{data\_index}\DUrole{o}{=}\DUrole{default_value}{0}}, \emph{\DUrole{n}{psf}\DUrole{o}{=}\DUrole{default_value}{None}}, \emph{\DUrole{n}{mom\_maps}\DUrole{o}{=}\DUrole{default_value}{None}}, \emph{\DUrole{n}{additional\_maps}\DUrole{o}{=}\DUrole{default_value}{{[}{]}}}, \emph{\DUrole{n}{center}\DUrole{o}{=}\DUrole{default_value}{None}}, \emph{\DUrole{n}{position\_angle}\DUrole{o}{=}\DUrole{default_value}{None}}, \emph{\DUrole{n}{eps}\DUrole{o}{=}\DUrole{default_value}{None}}, \emph{\DUrole{n}{vsys}\DUrole{o}{=}\DUrole{default_value}{0}}, \emph{\DUrole{n}{limits}\DUrole{o}{=}\DUrole{default_value}{None}}, \emph{\DUrole{n}{plot\_dir}\DUrole{o}{=}\DUrole{default_value}{None}}, \emph{\DUrole{o}{**}\DUrole{n}{kwargs}}}{}
\pysigstopsignatures
\sphinxAtStartPar
Bases: \sphinxcode{\sphinxupquote{object}}

\sphinxAtStartPar
Base input class for a CubeSPA object
\index{create\_spectra() (cubespa.cubespa.CubeSPA method)@\spxentry{create\_spectra()}\spxextra{cubespa.cubespa.CubeSPA method}}

\begin{fulllineitems}
\phantomsection\label{\detokenize{modules:cubespa.cubespa.CubeSPA.create_spectra}}
\pysigstartsignatures
\pysiglinewithargsret{\sphinxbfcode{\sphinxupquote{create\_spectra}}}{\emph{\DUrole{n}{position}}, \emph{\DUrole{n}{size}}, \emph{\DUrole{n}{return\_products}\DUrole{o}{=}\DUrole{default_value}{False}}, \emph{\DUrole{n}{plot}\DUrole{o}{=}\DUrole{default_value}{False}}}{}
\pysigstopsignatures
\end{fulllineitems}

\index{get\_beam\_area() (cubespa.cubespa.CubeSPA method)@\spxentry{get\_beam\_area()}\spxextra{cubespa.cubespa.CubeSPA method}}

\begin{fulllineitems}
\phantomsection\label{\detokenize{modules:cubespa.cubespa.CubeSPA.get_beam_area}}
\pysigstartsignatures
\pysiglinewithargsret{\sphinxbfcode{\sphinxupquote{get\_beam\_area}}}{}{}
\pysigstopsignatures
\end{fulllineitems}

\index{load\_dir() (cubespa.cubespa.CubeSPA method)@\spxentry{load\_dir()}\spxextra{cubespa.cubespa.CubeSPA method}}

\begin{fulllineitems}
\phantomsection\label{\detokenize{modules:cubespa.cubespa.CubeSPA.load_dir}}
\pysigstartsignatures
\pysiglinewithargsret{\sphinxbfcode{\sphinxupquote{load\_dir}}}{}{}
\pysigstopsignatures
\end{fulllineitems}

\index{plot\_moment\_maps() (cubespa.cubespa.CubeSPA method)@\spxentry{plot\_moment\_maps()}\spxextra{cubespa.cubespa.CubeSPA method}}

\begin{fulllineitems}
\phantomsection\label{\detokenize{modules:cubespa.cubespa.CubeSPA.plot_moment_maps}}
\pysigstartsignatures
\pysiglinewithargsret{\sphinxbfcode{\sphinxupquote{plot\_moment\_maps}}}{\emph{\DUrole{n}{use\_limits}\DUrole{o}{=}\DUrole{default_value}{True}}, \emph{\DUrole{o}{**}\DUrole{n}{kwargs}}}{}
\pysigstopsignatures
\end{fulllineitems}

\index{velocities\_from\_wcs() (cubespa.cubespa.CubeSPA method)@\spxentry{velocities\_from\_wcs()}\spxextra{cubespa.cubespa.CubeSPA method}}

\begin{fulllineitems}
\phantomsection\label{\detokenize{modules:cubespa.cubespa.CubeSPA.velocities_from_wcs}}
\pysigstartsignatures
\pysiglinewithargsret{\sphinxbfcode{\sphinxupquote{velocities\_from\_wcs}}}{\emph{\DUrole{n}{vsys}\DUrole{o}{=}\DUrole{default_value}{0}}}{}
\pysigstopsignatures
\end{fulllineitems}


\end{fulllineitems}

\phantomsection\label{\detokenize{modules:module-cubespa.data}}\index{module@\spxentry{module}!cubespa.data@\spxentry{cubespa.data}}\index{cubespa.data@\spxentry{cubespa.data}!module@\spxentry{module}}\index{DataSet (class in cubespa.data)@\spxentry{DataSet}\spxextra{class in cubespa.data}}

\begin{fulllineitems}
\phantomsection\label{\detokenize{modules:cubespa.data.DataSet}}
\pysigstartsignatures
\pysiglinewithargsret{\sphinxbfcode{\sphinxupquote{class\DUrole{w}{  }}}\sphinxcode{\sphinxupquote{cubespa.data.}}\sphinxbfcode{\sphinxupquote{DataSet}}}{\emph{\DUrole{n}{data}\DUrole{o}{=}\DUrole{default_value}{None}}, \emph{\DUrole{n}{wcs}\DUrole{o}{=}\DUrole{default_value}{None}}, \emph{\DUrole{n}{header}\DUrole{o}{=}\DUrole{default_value}{None}}, \emph{\DUrole{n}{label}\DUrole{o}{=}\DUrole{default_value}{None}}}{}
\pysigstopsignatures
\sphinxAtStartPar
Bases: \sphinxcode{\sphinxupquote{object}}

\end{fulllineitems}

\index{MomentMaps (class in cubespa.data)@\spxentry{MomentMaps}\spxextra{class in cubespa.data}}

\begin{fulllineitems}
\phantomsection\label{\detokenize{modules:cubespa.data.MomentMaps}}
\pysigstartsignatures
\pysiglinewithargsret{\sphinxbfcode{\sphinxupquote{class\DUrole{w}{  }}}\sphinxcode{\sphinxupquote{cubespa.data.}}\sphinxbfcode{\sphinxupquote{MomentMaps}}}{\emph{\DUrole{n}{mom0}\DUrole{o}{=}\DUrole{default_value}{None}}, \emph{\DUrole{n}{mom1}\DUrole{o}{=}\DUrole{default_value}{None}}, \emph{\DUrole{n}{mom2}\DUrole{o}{=}\DUrole{default_value}{None}}, \emph{\DUrole{n}{data\_index}\DUrole{o}{=}\DUrole{default_value}{0}}}{}
\pysigstopsignatures
\sphinxAtStartPar
Bases: \sphinxcode{\sphinxupquote{object}}

\end{fulllineitems}

\index{RGBImage (class in cubespa.data)@\spxentry{RGBImage}\spxextra{class in cubespa.data}}

\begin{fulllineitems}
\phantomsection\label{\detokenize{modules:cubespa.data.RGBImage}}
\pysigstartsignatures
\pysiglinewithargsret{\sphinxbfcode{\sphinxupquote{class\DUrole{w}{  }}}\sphinxcode{\sphinxupquote{cubespa.data.}}\sphinxbfcode{\sphinxupquote{RGBImage}}}{\emph{\DUrole{n}{data}\DUrole{o}{=}\DUrole{default_value}{None}}, \emph{\DUrole{n}{wcs}\DUrole{o}{=}\DUrole{default_value}{None}}, \emph{\DUrole{n}{header}\DUrole{o}{=}\DUrole{default_value}{None}}, \emph{\DUrole{n}{label}\DUrole{o}{=}\DUrole{default_value}{None}}}{}
\pysigstopsignatures
\sphinxAtStartPar
Bases: {\hyperref[\detokenize{modules:cubespa.data.DataSet}]{\sphinxcrossref{\sphinxcode{\sphinxupquote{DataSet}}}}}

\end{fulllineitems}

\index{gen\_cutout() (in module cubespa.data)@\spxentry{gen\_cutout()}\spxextra{in module cubespa.data}}

\begin{fulllineitems}
\phantomsection\label{\detokenize{modules:cubespa.data.gen_cutout}}
\pysigstartsignatures
\pysiglinewithargsret{\sphinxcode{\sphinxupquote{cubespa.data.}}\sphinxbfcode{\sphinxupquote{gen\_cutout}}}{\emph{\DUrole{n}{cubespa\_obj}}, \emph{\DUrole{n}{cent}}, \emph{\DUrole{n}{size}}, \emph{\DUrole{n}{show\_bbox}\DUrole{o}{=}\DUrole{default_value}{False}}}{}
\pysigstopsignatures
\end{fulllineitems}

\index{handle\_data() (in module cubespa.data)@\spxentry{handle\_data()}\spxextra{in module cubespa.data}}

\begin{fulllineitems}
\phantomsection\label{\detokenize{modules:cubespa.data.handle_data}}
\pysigstartsignatures
\pysiglinewithargsret{\sphinxcode{\sphinxupquote{cubespa.data.}}\sphinxbfcode{\sphinxupquote{handle\_data}}}{\emph{\DUrole{n}{data}}, \emph{\DUrole{n}{handler}}, \emph{\DUrole{n}{data\_index}\DUrole{o}{=}\DUrole{default_value}{0}}}{}
\pysigstopsignatures
\sphinxAtStartPar
Handle incoming data for the CubeSPA object.
If not a string, CubeSPA will return the data enclosed in a dataset without additional info.
:param data: Incoming data. If str, CubeSPA will automatically load it in.
:type data: str or cubespa.DataSet
:param handler: Method to handle data, either set to load\_data or load\_moment\_maps
:type handler: method
:param data\_index: Index to find data in. Defaults to 0.
:type data\_index: int, optional
\begin{quote}\begin{description}
\sphinxlineitem{Returns}
\sphinxAtStartPar
The output DataSet object.

\sphinxlineitem{Return type}
\sphinxAtStartPar
utils.DataSet

\end{description}\end{quote}

\end{fulllineitems}

\index{load\_data() (in module cubespa.data)@\spxentry{load\_data()}\spxextra{in module cubespa.data}}

\begin{fulllineitems}
\phantomsection\label{\detokenize{modules:cubespa.data.load_data}}
\pysigstartsignatures
\pysiglinewithargsret{\sphinxcode{\sphinxupquote{cubespa.data.}}\sphinxbfcode{\sphinxupquote{load\_data}}}{\emph{\DUrole{n}{filename}}, \emph{\DUrole{n}{data\_index}\DUrole{o}{=}\DUrole{default_value}{0}}, \emph{\DUrole{n}{rgb\_index}\DUrole{o}{=}\DUrole{default_value}{None}}, \emph{\DUrole{n}{label}\DUrole{o}{=}\DUrole{default_value}{None}}}{}
\pysigstopsignatures
\end{fulllineitems}

\index{load\_moment\_maps() (in module cubespa.data)@\spxentry{load\_moment\_maps()}\spxextra{in module cubespa.data}}

\begin{fulllineitems}
\phantomsection\label{\detokenize{modules:cubespa.data.load_moment_maps}}
\pysigstartsignatures
\pysiglinewithargsret{\sphinxcode{\sphinxupquote{cubespa.data.}}\sphinxbfcode{\sphinxupquote{load\_moment\_maps}}}{\emph{\DUrole{n}{topdir}}, \emph{\DUrole{n}{data\_index}\DUrole{o}{=}\DUrole{default_value}{0}}}{}
\pysigstopsignatures
\end{fulllineitems}

\phantomsection\label{\detokenize{modules:module-cubespa.spectra}}\index{module@\spxentry{module}!cubespa.spectra@\spxentry{cubespa.spectra}}\index{cubespa.spectra@\spxentry{cubespa.spectra}!module@\spxentry{module}}\index{align\_apertures() (in module cubespa.spectra)@\spxentry{align\_apertures()}\spxextra{in module cubespa.spectra}}

\begin{fulllineitems}
\phantomsection\label{\detokenize{modules:cubespa.spectra.align_apertures}}
\pysigstartsignatures
\pysiglinewithargsret{\sphinxcode{\sphinxupquote{cubespa.spectra.}}\sphinxbfcode{\sphinxupquote{align\_apertures}}}{\emph{\DUrole{n}{aper\_list}}, \emph{\DUrole{n}{wcs1}}, \emph{\DUrole{n}{wcs2}}}{}
\pysigstopsignatures
\sphinxAtStartPar
Align and resize a set of apertures from wcs1 to wcs2
\begin{quote}\begin{description}
\sphinxlineitem{Parameters}\begin{itemize}
\item {} 
\sphinxAtStartPar
\sphinxstyleliteralstrong{\sphinxupquote{aper\_list}} (\sphinxstyleliteralemphasis{\sphinxupquote{list}}) \textendash{} List of apertures in the following format:
{[}(p1, p2), (s1, s2){]} where the first tuple is the (x,y) position and the second tuple is the (x,y) height.

\item {} 
\sphinxAtStartPar
\sphinxstyleliteralstrong{\sphinxupquote{wcs1}} (\sphinxstyleliteralemphasis{\sphinxupquote{astropy.wcs.WCS}}) \textendash{} WCS that aper\_list apertures are already aligned to

\item {} 
\sphinxAtStartPar
\sphinxstyleliteralstrong{\sphinxupquote{wcs2}} (\sphinxstyleliteralemphasis{\sphinxupquote{astropy.wcs.WCS}}) \textendash{} WCS to align apertures to

\end{itemize}

\sphinxlineitem{Returns}
\sphinxAtStartPar
\_description\_

\sphinxlineitem{Return type}
\sphinxAtStartPar
\_type\_

\end{description}\end{quote}

\end{fulllineitems}

\index{analyze\_spectra() (in module cubespa.spectra)@\spxentry{analyze\_spectra()}\spxextra{in module cubespa.spectra}}

\begin{fulllineitems}
\phantomsection\label{\detokenize{modules:cubespa.spectra.analyze_spectra}}
\pysigstartsignatures
\pysiglinewithargsret{\sphinxcode{\sphinxupquote{cubespa.spectra.}}\sphinxbfcode{\sphinxupquote{analyze\_spectra}}}{\emph{\DUrole{n}{spec}}, \emph{\DUrole{n}{sigma}\DUrole{o}{=}\DUrole{default_value}{2}}, \emph{\DUrole{n}{cmin}\DUrole{o}{=}\DUrole{default_value}{None}}, \emph{\DUrole{n}{cmax}\DUrole{o}{=}\DUrole{default_value}{None}}}{}
\pysigstopsignatures
\end{fulllineitems}

\index{calc\_snr() (in module cubespa.spectra)@\spxentry{calc\_snr()}\spxextra{in module cubespa.spectra}}

\begin{fulllineitems}
\phantomsection\label{\detokenize{modules:cubespa.spectra.calc_snr}}
\pysigstartsignatures
\pysiglinewithargsret{\sphinxcode{\sphinxupquote{cubespa.spectra.}}\sphinxbfcode{\sphinxupquote{calc\_snr}}}{\emph{\DUrole{n}{spec}}, \emph{\DUrole{n}{chan\_min}}, \emph{\DUrole{n}{chan\_max}}}{}
\pysigstopsignatures
\end{fulllineitems}

\index{create\_aperture() (in module cubespa.spectra)@\spxentry{create\_aperture()}\spxextra{in module cubespa.spectra}}

\begin{fulllineitems}
\phantomsection\label{\detokenize{modules:cubespa.spectra.create_aperture}}
\pysigstartsignatures
\pysiglinewithargsret{\sphinxcode{\sphinxupquote{cubespa.spectra.}}\sphinxbfcode{\sphinxupquote{create\_aperture}}}{\emph{\DUrole{n}{cubespa\_obj}}, \emph{\DUrole{n}{position}}, \emph{\DUrole{n}{shape}}, \emph{\DUrole{n}{aper\_type}\DUrole{o}{=}\DUrole{default_value}{\textquotesingle{}elliptical\textquotesingle{}}}, \emph{\DUrole{n}{plot}\DUrole{o}{=}\DUrole{default_value}{False}}}{}
\pysigstopsignatures
\sphinxAtStartPar
Generate photutils aperture of desired type, position, and shape.
\begin{quote}\begin{description}
\sphinxlineitem{Returns}
\sphinxAtStartPar
Photutils aperture

\sphinxlineitem{Return type}
\sphinxAtStartPar
photutils.aperture

\end{description}\end{quote}

\end{fulllineitems}

\index{get\_spectra() (in module cubespa.spectra)@\spxentry{get\_spectra()}\spxextra{in module cubespa.spectra}}

\begin{fulllineitems}
\phantomsection\label{\detokenize{modules:cubespa.spectra.get_spectra}}
\pysigstartsignatures
\pysiglinewithargsret{\sphinxcode{\sphinxupquote{cubespa.spectra.}}\sphinxbfcode{\sphinxupquote{get\_spectra}}}{\emph{\DUrole{n}{cube}}, \emph{\DUrole{n}{aper}}}{}
\pysigstopsignatures
\sphinxAtStartPar
Get the spectra through a datacube at the position and size of a given aperture.
\begin{quote}\begin{description}
\sphinxlineitem{Parameters}\begin{itemize}
\item {} 
\sphinxAtStartPar
\sphinxstyleliteralstrong{\sphinxupquote{cube}} (\sphinxstyleliteralemphasis{\sphinxupquote{ndarray}}) \textendash{} \_description\_

\item {} 
\sphinxAtStartPar
\sphinxstyleliteralstrong{\sphinxupquote{aper}} (\sphinxstyleliteralemphasis{\sphinxupquote{photutils aperture}}) \textendash{} Elliptical or circular aperture/annulus.

\end{itemize}

\sphinxlineitem{Returns}
\sphinxAtStartPar
\_description\_

\sphinxlineitem{Return type}
\sphinxAtStartPar
\_type\_

\end{description}\end{quote}

\end{fulllineitems}

\index{multi\_spec() (in module cubespa.spectra)@\spxentry{multi\_spec()}\spxextra{in module cubespa.spectra}}

\begin{fulllineitems}
\phantomsection\label{\detokenize{modules:cubespa.spectra.multi_spec}}
\pysigstartsignatures
\pysiglinewithargsret{\sphinxcode{\sphinxupquote{cubespa.spectra.}}\sphinxbfcode{\sphinxupquote{multi\_spec}}}{\emph{\DUrole{n}{cubespa\_obj}}, \emph{\DUrole{n}{spec\_info}}}{}
\pysigstopsignatures
\sphinxAtStartPar
Generate a list of apertures and spectra for better diagnostics.
\begin{quote}\begin{description}
\sphinxlineitem{Parameters}\begin{itemize}
\item {} 
\sphinxAtStartPar
\sphinxstyleliteralstrong{\sphinxupquote{cubespa\_obj}} ({\hyperref[\detokenize{modules:cubespa.cubespa.CubeSPA}]{\sphinxcrossref{\sphinxstyleliteralemphasis{\sphinxupquote{cubespa.CubeSPA}}}}}) \textendash{} Input CubeSPA object to pull spectra from.

\item {} 
\sphinxAtStartPar
\sphinxstyleliteralstrong{\sphinxupquote{spec\_info}} (\sphinxstyleliteralemphasis{\sphinxupquote{\_type\_}}) \textendash{} A list of spectra position and sizes, for example, to get two spectra, you would

\item {} 
\sphinxAtStartPar
\sphinxstyleliteralstrong{\sphinxupquote{have}} \textendash{} \begin{description}
\sphinxlineitem{spec\_info = {[}{[}(p1, p2), (s1, s2){]},}
\sphinxAtStartPar
{[}(p3, p4), (s3, s4){]}{]}

\end{description}


\end{itemize}

\sphinxlineitem{Returns}
\sphinxAtStartPar
A list of apertures and a list of spectra.

\sphinxlineitem{Return type}
\sphinxAtStartPar
tuple(array)

\end{description}\end{quote}

\end{fulllineitems}

\index{spectra\_comparison() (in module cubespa.spectra)@\spxentry{spectra\_comparison()}\spxextra{in module cubespa.spectra}}

\begin{fulllineitems}
\phantomsection\label{\detokenize{modules:cubespa.spectra.spectra_comparison}}
\pysigstartsignatures
\pysiglinewithargsret{\sphinxcode{\sphinxupquote{cubespa.spectra.}}\sphinxbfcode{\sphinxupquote{spectra\_comparison}}}{\emph{\DUrole{n}{cubecomp}}, \emph{\DUrole{n}{aper\_list}}, \emph{\DUrole{n}{outname}\DUrole{o}{=}\DUrole{default_value}{None}}, \emph{\DUrole{n}{plot\_ticks}\DUrole{o}{=}\DUrole{default_value}{True}}, \emph{\DUrole{n}{chan\_ranges}\DUrole{o}{=}\DUrole{default_value}{None}}}{}
\pysigstopsignatures
\end{fulllineitems}

\phantomsection\label{\detokenize{modules:module-cubespa.utils}}\index{module@\spxentry{module}!cubespa.utils@\spxentry{cubespa.utils}}\index{cubespa.utils@\spxentry{cubespa.utils}!module@\spxentry{module}}\index{H2\_Mass() (in module cubespa.utils)@\spxentry{H2\_Mass()}\spxextra{in module cubespa.utils}}

\begin{fulllineitems}
\phantomsection\label{\detokenize{modules:cubespa.utils.H2_Mass}}
\pysigstartsignatures
\pysiglinewithargsret{\sphinxcode{\sphinxupquote{cubespa.utils.}}\sphinxbfcode{\sphinxupquote{H2\_Mass}}}{\emph{\DUrole{n}{SCO}}, \emph{\DUrole{n}{D\_L}\DUrole{o}{=}\DUrole{default_value}{100.0}}, \emph{\DUrole{n}{z\_gal}\DUrole{o}{=}\DUrole{default_value}{0.01}}, \emph{\DUrole{n}{freq}\DUrole{o}{=}\DUrole{default_value}{220.0}}, \emph{\DUrole{n}{a\_CO}\DUrole{o}{=}\DUrole{default_value}{3.2}}, \emph{\DUrole{n}{R\_21}\DUrole{o}{=}\DUrole{default_value}{0.8}}}{}
\pysigstopsignatures
\sphinxAtStartPar
Calculate the H2 mass from a CO(2\sphinxhyphen{}1) emission map
\begin{quote}\begin{description}
\sphinxlineitem{Parameters}\begin{itemize}
\item {} 
\sphinxAtStartPar
\sphinxstyleliteralstrong{\sphinxupquote{SCO}} (\sphinxstyleliteralemphasis{\sphinxupquote{float}}) \textendash{} The integrated CO(2\sphinxhyphen{}1) flux (in Jy km/s)

\item {} 
\sphinxAtStartPar
\sphinxstyleliteralstrong{\sphinxupquote{D\_L}} (\sphinxstyleliteralemphasis{\sphinxupquote{float}}\sphinxstyleliteralemphasis{\sphinxupquote{, }}\sphinxstyleliteralemphasis{\sphinxupquote{optional}}) \textendash{} The luminosity distance (in MPC). Defaults to 100.

\item {} 
\sphinxAtStartPar
\sphinxstyleliteralstrong{\sphinxupquote{z\_gal}} (\sphinxstyleliteralemphasis{\sphinxupquote{float}}\sphinxstyleliteralemphasis{\sphinxupquote{, }}\sphinxstyleliteralemphasis{\sphinxupquote{optional}}) \textendash{} The redshift of the galaxy. Defaults to 0.01.

\item {} 
\sphinxAtStartPar
\sphinxstyleliteralstrong{\sphinxupquote{freq}} (\sphinxstyleliteralemphasis{\sphinxupquote{float}}\sphinxstyleliteralemphasis{\sphinxupquote{, }}\sphinxstyleliteralemphasis{\sphinxupquote{optional}}) \textendash{} The observing frequency (in GHz). Defaults to 220.

\item {} 
\sphinxAtStartPar
\sphinxstyleliteralstrong{\sphinxupquote{a\_CO}} (\sphinxstyleliteralemphasis{\sphinxupquote{float}}\sphinxstyleliteralemphasis{\sphinxupquote{, }}\sphinxstyleliteralemphasis{\sphinxupquote{optional}}) \textendash{} The CO conversion factor (in solar masses per square pc). Defaults to 3.2.

\item {} 
\sphinxAtStartPar
\sphinxstyleliteralstrong{\sphinxupquote{R\_21}} (\sphinxstyleliteralemphasis{\sphinxupquote{float}}\sphinxstyleliteralemphasis{\sphinxupquote{, }}\sphinxstyleliteralemphasis{\sphinxupquote{optional}}) \textendash{} The CO(2\sphinxhyphen{}1)/(1\sphinxhyphen{}0) ratio. Defaults to 0.8 (from Leroy et al, 2009).

\end{itemize}

\end{description}\end{quote}

\end{fulllineitems}

\index{RMS() (in module cubespa.utils)@\spxentry{RMS()}\spxextra{in module cubespa.utils}}

\begin{fulllineitems}
\phantomsection\label{\detokenize{modules:cubespa.utils.RMS}}
\pysigstartsignatures
\pysiglinewithargsret{\sphinxcode{\sphinxupquote{cubespa.utils.}}\sphinxbfcode{\sphinxupquote{RMS}}}{\emph{\DUrole{n}{a}}, \emph{\DUrole{n}{sigclip}\DUrole{o}{=}\DUrole{default_value}{False}}}{}
\pysigstopsignatures
\end{fulllineitems}

\index{beam\_area() (in module cubespa.utils)@\spxentry{beam\_area()}\spxextra{in module cubespa.utils}}

\begin{fulllineitems}
\phantomsection\label{\detokenize{modules:cubespa.utils.beam_area}}
\pysigstartsignatures
\pysiglinewithargsret{\sphinxcode{\sphinxupquote{cubespa.utils.}}\sphinxbfcode{\sphinxupquote{beam\_area}}}{\emph{\DUrole{n}{bmaj}}, \emph{\DUrole{n}{bmin}}}{}
\pysigstopsignatures\begin{description}
\sphinxlineitem{Get the area of the beam based on a standard elliptical Gaussian beam with major and}
\sphinxAtStartPar
minor axes bmaj and bmin.

\end{description}
\begin{quote}\begin{description}
\sphinxlineitem{Parameters}\begin{itemize}
\item {} 
\sphinxAtStartPar
\sphinxstyleliteralstrong{\sphinxupquote{bmaj}} (\sphinxstyleliteralemphasis{\sphinxupquote{float}}) \textendash{} Major axis of beam

\item {} 
\sphinxAtStartPar
\sphinxstyleliteralstrong{\sphinxupquote{bmin}} (\sphinxstyleliteralemphasis{\sphinxupquote{float}}) \textendash{} Minor axis of beam

\end{itemize}

\sphinxlineitem{Returns}
\sphinxAtStartPar
The beam area.

\sphinxlineitem{Return type}
\sphinxAtStartPar
float

\end{description}\end{quote}

\end{fulllineitems}

\index{bounds\_from\_moment\_map() (in module cubespa.utils)@\spxentry{bounds\_from\_moment\_map()}\spxextra{in module cubespa.utils}}

\begin{fulllineitems}
\phantomsection\label{\detokenize{modules:cubespa.utils.bounds_from_moment_map}}
\pysigstartsignatures
\pysiglinewithargsret{\sphinxcode{\sphinxupquote{cubespa.utils.}}\sphinxbfcode{\sphinxupquote{bounds\_from\_moment\_map}}}{\emph{\DUrole{n}{data}}, \emph{\DUrole{n}{padding}\DUrole{o}{=}\DUrole{default_value}{0}}}{}
\pysigstopsignatures
\end{fulllineitems}

\index{centre\_coords() (in module cubespa.utils)@\spxentry{centre\_coords()}\spxextra{in module cubespa.utils}}

\begin{fulllineitems}
\phantomsection\label{\detokenize{modules:cubespa.utils.centre_coords}}
\pysigstartsignatures
\pysiglinewithargsret{\sphinxcode{\sphinxupquote{cubespa.utils.}}\sphinxbfcode{\sphinxupquote{centre\_coords}}}{\emph{\DUrole{n}{input\_wcs}}, \emph{\DUrole{n}{ra}}, \emph{\DUrole{n}{dec}}}{}
\pysigstopsignatures
\end{fulllineitems}

\index{check\_and\_make\_dir() (in module cubespa.utils)@\spxentry{check\_and\_make\_dir()}\spxextra{in module cubespa.utils}}

\begin{fulllineitems}
\phantomsection\label{\detokenize{modules:cubespa.utils.check_and_make_dir}}
\pysigstartsignatures
\pysiglinewithargsret{\sphinxcode{\sphinxupquote{cubespa.utils.}}\sphinxbfcode{\sphinxupquote{check\_and\_make\_dir}}}{\emph{\DUrole{n}{directory}}}{}
\pysigstopsignatures
\end{fulllineitems}

\index{check\_kwarg() (in module cubespa.utils)@\spxentry{check\_kwarg()}\spxextra{in module cubespa.utils}}

\begin{fulllineitems}
\phantomsection\label{\detokenize{modules:cubespa.utils.check_kwarg}}
\pysigstartsignatures
\pysiglinewithargsret{\sphinxcode{\sphinxupquote{cubespa.utils.}}\sphinxbfcode{\sphinxupquote{check\_kwarg}}}{\emph{\DUrole{n}{key}}, \emph{\DUrole{n}{default}}, \emph{\DUrole{n}{kwargs}\DUrole{p}{:}\DUrole{w}{  }\DUrole{n}{dict}}}{}
\pysigstopsignatures
\end{fulllineitems}

\index{create\_channel\_ranges() (in module cubespa.utils)@\spxentry{create\_channel\_ranges()}\spxextra{in module cubespa.utils}}

\begin{fulllineitems}
\phantomsection\label{\detokenize{modules:cubespa.utils.create_channel_ranges}}
\pysigstartsignatures
\pysiglinewithargsret{\sphinxcode{\sphinxupquote{cubespa.utils.}}\sphinxbfcode{\sphinxupquote{create\_channel\_ranges}}}{\emph{\DUrole{n}{n}}}{}
\pysigstopsignatures
\sphinxAtStartPar
Create an array of None values of length n (to initialize channel ranges for spectral analysis)

\end{fulllineitems}

\index{estimate\_rms() (in module cubespa.utils)@\spxentry{estimate\_rms()}\spxextra{in module cubespa.utils}}

\begin{fulllineitems}
\phantomsection\label{\detokenize{modules:cubespa.utils.estimate_rms}}
\pysigstartsignatures
\pysiglinewithargsret{\sphinxcode{\sphinxupquote{cubespa.utils.}}\sphinxbfcode{\sphinxupquote{estimate\_rms}}}{\emph{\DUrole{n}{cube}}, \emph{\DUrole{n}{channel\_min}}, \emph{\DUrole{n}{channel\_max}}}{}
\pysigstopsignatures
\end{fulllineitems}

\index{im\_bounds() (in module cubespa.utils)@\spxentry{im\_bounds()}\spxextra{in module cubespa.utils}}

\begin{fulllineitems}
\phantomsection\label{\detokenize{modules:cubespa.utils.im_bounds}}
\pysigstartsignatures
\pysiglinewithargsret{\sphinxcode{\sphinxupquote{cubespa.utils.}}\sphinxbfcode{\sphinxupquote{im\_bounds}}}{\emph{\DUrole{n}{stats}}, \emph{\DUrole{n}{sigma}\DUrole{o}{=}\DUrole{default_value}{1}}}{}
\pysigstopsignatures
\end{fulllineitems}

\index{imstat() (in module cubespa.utils)@\spxentry{imstat()}\spxextra{in module cubespa.utils}}

\begin{fulllineitems}
\phantomsection\label{\detokenize{modules:cubespa.utils.imstat}}
\pysigstartsignatures
\pysiglinewithargsret{\sphinxcode{\sphinxupquote{cubespa.utils.}}\sphinxbfcode{\sphinxupquote{imstat}}}{\emph{\DUrole{n}{a}}}{}
\pysigstopsignatures
\sphinxAtStartPar
Simple tool to return statistics for a given array.
\begin{quote}\begin{description}
\sphinxlineitem{Parameters}
\sphinxAtStartPar
\sphinxstyleliteralstrong{\sphinxupquote{a}} (\sphinxstyleliteralemphasis{\sphinxupquote{ndarray}}) \textendash{} Input array to get statistics.

\sphinxlineitem{Returns}
\sphinxAtStartPar
\_description\_

\sphinxlineitem{Return type}
\sphinxAtStartPar
\_type\_

\end{description}\end{quote}

\end{fulllineitems}

\index{line\_corners() (in module cubespa.utils)@\spxentry{line\_corners()}\spxextra{in module cubespa.utils}}

\begin{fulllineitems}
\phantomsection\label{\detokenize{modules:cubespa.utils.line_corners}}
\pysigstartsignatures
\pysiglinewithargsret{\sphinxcode{\sphinxupquote{cubespa.utils.}}\sphinxbfcode{\sphinxupquote{line\_corners}}}{\emph{\DUrole{n}{x0}}, \emph{\DUrole{n}{y0}}, \emph{\DUrole{n}{w}}, \emph{\DUrole{n}{L}}, \emph{\DUrole{n}{theta}}}{}
\pysigstopsignatures
\end{fulllineitems}

\index{line\_endpoints() (in module cubespa.utils)@\spxentry{line\_endpoints()}\spxextra{in module cubespa.utils}}

\begin{fulllineitems}
\phantomsection\label{\detokenize{modules:cubespa.utils.line_endpoints}}
\pysigstartsignatures
\pysiglinewithargsret{\sphinxcode{\sphinxupquote{cubespa.utils.}}\sphinxbfcode{\sphinxupquote{line\_endpoints}}}{\emph{\DUrole{n}{x0}}, \emph{\DUrole{n}{y0}}, \emph{\DUrole{n}{L}}, \emph{\DUrole{n}{theta}}}{}
\pysigstopsignatures
\end{fulllineitems}

\index{match\_wcs\_axes() (in module cubespa.utils)@\spxentry{match\_wcs\_axes()}\spxextra{in module cubespa.utils}}

\begin{fulllineitems}
\phantomsection\label{\detokenize{modules:cubespa.utils.match_wcs_axes}}
\pysigstartsignatures
\pysiglinewithargsret{\sphinxcode{\sphinxupquote{cubespa.utils.}}\sphinxbfcode{\sphinxupquote{match\_wcs\_axes}}}{\emph{\DUrole{n}{wcs1}}, \emph{\DUrole{n}{wcs2}}}{}
\pysigstopsignatures\begin{description}
\sphinxlineitem{Match the axes in WCS axes (mostly for image alignment with cubes and images).}
\sphinxAtStartPar
This method assumes that the ra/dec axes are always at indices 0 and 1.

\end{description}
\begin{quote}\begin{description}
\sphinxlineitem{Parameters}\begin{itemize}
\item {} 
\sphinxAtStartPar
\sphinxstyleliteralstrong{\sphinxupquote{wcs1}} (\sphinxstyleliteralemphasis{\sphinxupquote{astropy.wcs}}) \textendash{} WCS A

\item {} 
\sphinxAtStartPar
\sphinxstyleliteralstrong{\sphinxupquote{wcs2}} (\sphinxstyleliteralemphasis{\sphinxupquote{astropy.wcs}}) \textendash{} WCS B

\end{itemize}

\sphinxlineitem{Returns}
\sphinxAtStartPar
Both WCS objects.

\sphinxlineitem{Return type}
\sphinxAtStartPar
\_type\_

\end{description}\end{quote}

\end{fulllineitems}

\index{normalize() (in module cubespa.utils)@\spxentry{normalize()}\spxextra{in module cubespa.utils}}

\begin{fulllineitems}
\phantomsection\label{\detokenize{modules:cubespa.utils.normalize}}
\pysigstartsignatures
\pysiglinewithargsret{\sphinxcode{\sphinxupquote{cubespa.utils.}}\sphinxbfcode{\sphinxupquote{normalize}}}{\emph{\DUrole{n}{a}}, \emph{\DUrole{n}{clip\_low}\DUrole{o}{=}\DUrole{default_value}{None}}, \emph{\DUrole{n}{clip\_high}\DUrole{o}{=}\DUrole{default_value}{None}}, \emph{\DUrole{n}{stretch}\DUrole{o}{=}\DUrole{default_value}{None}}}{}
\pysigstopsignatures
\end{fulllineitems}

\index{normalized\_rgb\_image() (in module cubespa.utils)@\spxentry{normalized\_rgb\_image()}\spxextra{in module cubespa.utils}}

\begin{fulllineitems}
\phantomsection\label{\detokenize{modules:cubespa.utils.normalized_rgb_image}}
\pysigstartsignatures
\pysiglinewithargsret{\sphinxcode{\sphinxupquote{cubespa.utils.}}\sphinxbfcode{\sphinxupquote{normalized\_rgb\_image}}}{\emph{\DUrole{n}{image}}, \emph{\DUrole{n}{sigma}\DUrole{o}{=}\DUrole{default_value}{1}}, \emph{\DUrole{n}{stretch}\DUrole{o}{=}\DUrole{default_value}{None}}}{}
\pysigstopsignatures
\sphinxAtStartPar
Generate a properly formatted RGB image from a 3xmxn input.
\begin{quote}\begin{description}
\sphinxlineitem{Parameters}\begin{itemize}
\item {} 
\sphinxAtStartPar
\sphinxstyleliteralstrong{\sphinxupquote{image}} (\sphinxstyleliteralemphasis{\sphinxupquote{ndarray}}) \textendash{} 3 x m x n input image.

\item {} 
\sphinxAtStartPar
\sphinxstyleliteralstrong{\sphinxupquote{sigma}} (\sphinxstyleliteralemphasis{\sphinxupquote{int}}\sphinxstyleliteralemphasis{\sphinxupquote{, }}\sphinxstyleliteralemphasis{\sphinxupquote{float}}\sphinxstyleliteralemphasis{\sphinxupquote{, or }}\sphinxstyleliteralemphasis{\sphinxupquote{array optional}}) \textendash{} Sigma levels to adjust stretches and clips. Defaults to 1.
int: sigma for all 3 RGB images
tuple: All 3 images clipped to (sigma{[}0{]}, sigma{[}1{]})
Array of len(3): R,G,B images clipped to sigma{[}0{]}, sigma{[}1{]}, sigma{[}2{]}, respectively.

\item {} 
\sphinxAtStartPar
\sphinxstyleliteralstrong{\sphinxupquote{stretch}} (\sphinxstyleliteralemphasis{\sphinxupquote{str}}\sphinxstyleliteralemphasis{\sphinxupquote{, }}\sphinxstyleliteralemphasis{\sphinxupquote{optional}}) \textendash{} Type of stretch to apply. Defaults to None.

\end{itemize}

\sphinxlineitem{Returns}
\sphinxAtStartPar
The properly transposed, stretched and clipped RGB image.

\sphinxlineitem{Return type}
\sphinxAtStartPar
ndarray

\end{description}\end{quote}

\end{fulllineitems}

\index{recommended\_figsize() (in module cubespa.utils)@\spxentry{recommended\_figsize()}\spxextra{in module cubespa.utils}}

\begin{fulllineitems}
\phantomsection\label{\detokenize{modules:cubespa.utils.recommended_figsize}}
\pysigstartsignatures
\pysiglinewithargsret{\sphinxcode{\sphinxupquote{cubespa.utils.}}\sphinxbfcode{\sphinxupquote{recommended\_figsize}}}{\emph{\DUrole{n}{a}}, \emph{\DUrole{n}{width}\DUrole{o}{=}\DUrole{default_value}{8}}}{}
\pysigstopsignatures
\end{fulllineitems}



\subsection{Plotting}
\label{\detokenize{modules:module-cubespa.plotting.channel_map_plots}}\label{\detokenize{modules:plotting}}\index{module@\spxentry{module}!cubespa.plotting.channel\_map\_plots@\spxentry{cubespa.plotting.channel\_map\_plots}}\index{cubespa.plotting.channel\_map\_plots@\spxentry{cubespa.plotting.channel\_map\_plots}!module@\spxentry{module}}\index{channel\_maps() (in module cubespa.plotting.channel\_map\_plots)@\spxentry{channel\_maps()}\spxextra{in module cubespa.plotting.channel\_map\_plots}}

\begin{fulllineitems}
\phantomsection\label{\detokenize{modules:cubespa.plotting.channel_map_plots.channel_maps}}
\pysigstartsignatures
\pysiglinewithargsret{\sphinxcode{\sphinxupquote{cubespa.plotting.channel\_map\_plots.}}\sphinxbfcode{\sphinxupquote{channel\_maps}}}{\emph{\DUrole{n}{cubespa\_obj}}, \emph{\DUrole{n}{n\_chan}\DUrole{o}{=}\DUrole{default_value}{25}}, \emph{\DUrole{n}{limits}\DUrole{o}{=}\DUrole{default_value}{None}}, \emph{\DUrole{o}{**}\DUrole{n}{kwargs}}}{}
\pysigstopsignatures
\end{fulllineitems}

\phantomsection\label{\detokenize{modules:module-cubespa.plotting.mommap_plots}}\index{module@\spxentry{module}!cubespa.plotting.mommap\_plots@\spxentry{cubespa.plotting.mommap\_plots}}\index{cubespa.plotting.mommap\_plots@\spxentry{cubespa.plotting.mommap\_plots}!module@\spxentry{module}}\index{moment\_map\_plot() (in module cubespa.plotting.mommap\_plots)@\spxentry{moment\_map\_plot()}\spxextra{in module cubespa.plotting.mommap\_plots}}

\begin{fulllineitems}
\phantomsection\label{\detokenize{modules:cubespa.plotting.mommap_plots.moment_map_plot}}
\pysigstartsignatures
\pysiglinewithargsret{\sphinxcode{\sphinxupquote{cubespa.plotting.mommap\_plots.}}\sphinxbfcode{\sphinxupquote{moment\_map\_plot}}}{\emph{\DUrole{n}{cubespa\_obj}}, \emph{\DUrole{n}{filename}\DUrole{o}{=}\DUrole{default_value}{None}}, \emph{\DUrole{n}{use\_limits}\DUrole{o}{=}\DUrole{default_value}{True}}, \emph{\DUrole{o}{**}\DUrole{n}{kwargs}}}{}
\pysigstopsignatures
\sphinxAtStartPar
Generate moment map plots.
\begin{quote}\begin{description}
\sphinxlineitem{Parameters}\begin{itemize}
\item {} 
\sphinxAtStartPar
\sphinxstyleliteralstrong{\sphinxupquote{cubespa\_obj}} ({\hyperref[\detokenize{modules:cubespa.cubespa.CubeSPA}]{\sphinxcrossref{\sphinxstyleliteralemphasis{\sphinxupquote{cubespa.CubeSPA}}}}}) \textendash{} The input CubeSPA object, with valid moment maps loaded.

\item {} 
\sphinxAtStartPar
\sphinxstyleliteralstrong{\sphinxupquote{filename}} (\sphinxstyleliteralemphasis{\sphinxupquote{str}}\sphinxstyleliteralemphasis{\sphinxupquote{, }}\sphinxstyleliteralemphasis{\sphinxupquote{optional}}) \textendash{} Output filename, in which the plot is just shown instead of saved. Defaults to None.

\item {} 
\sphinxAtStartPar
\sphinxstyleliteralstrong{\sphinxupquote{use\_limits}} (\sphinxstyleliteralemphasis{\sphinxupquote{bool}}\sphinxstyleliteralemphasis{\sphinxupquote{, }}\sphinxstyleliteralemphasis{\sphinxupquote{optional}}) \textendash{} 
\sphinxAtStartPar
Whether or not to use limits from the CubeSPA object.

\sphinxAtStartPar
It is a good idea to set to False for cutout objects, as their limits will be relative to the initial CubeSPA object,
and their desired limits will be the entire array. Defaults to True.


\end{itemize}

\end{description}\end{quote}

\end{fulllineitems}

\phantomsection\label{\detokenize{modules:module-cubespa.plotting.overlay_plots}}\index{module@\spxentry{module}!cubespa.plotting.overlay\_plots@\spxentry{cubespa.plotting.overlay\_plots}}\index{cubespa.plotting.overlay\_plots@\spxentry{cubespa.plotting.overlay\_plots}!module@\spxentry{module}}\index{overlay\_plot() (in module cubespa.plotting.overlay\_plots)@\spxentry{overlay\_plot()}\spxextra{in module cubespa.plotting.overlay\_plots}}

\begin{fulllineitems}
\phantomsection\label{\detokenize{modules:cubespa.plotting.overlay_plots.overlay_plot}}
\pysigstartsignatures
\pysiglinewithargsret{\sphinxcode{\sphinxupquote{cubespa.plotting.overlay\_plots.}}\sphinxbfcode{\sphinxupquote{overlay\_plot}}}{\emph{\DUrole{n}{img\_obj}}, \emph{\DUrole{n}{overlay\_obj}}, \emph{\DUrole{n}{lims}\DUrole{o}{=}\DUrole{default_value}{None}}, \emph{\DUrole{o}{**}\DUrole{n}{kwargs}}}{}
\pysigstopsignatures
\end{fulllineitems}

\index{plot\_psf\_overlay() (in module cubespa.plotting.overlay\_plots)@\spxentry{plot\_psf\_overlay()}\spxextra{in module cubespa.plotting.overlay\_plots}}

\begin{fulllineitems}
\phantomsection\label{\detokenize{modules:cubespa.plotting.overlay_plots.plot_psf_overlay}}
\pysigstartsignatures
\pysiglinewithargsret{\sphinxcode{\sphinxupquote{cubespa.plotting.overlay\_plots.}}\sphinxbfcode{\sphinxupquote{plot\_psf\_overlay}}}{\emph{\DUrole{n}{cubespa\_obj}}, \emph{\DUrole{n}{psf\_conv}\DUrole{o}{=}\DUrole{default_value}{None}}, \emph{\DUrole{n}{x0}\DUrole{o}{=}\DUrole{default_value}{0}}, \emph{\DUrole{n}{y0}\DUrole{o}{=}\DUrole{default_value}{0}}, \emph{\DUrole{o}{**}\DUrole{n}{kwargs}}}{}
\pysigstopsignatures
\end{fulllineitems}

\index{rgb\_overlay() (in module cubespa.plotting.overlay\_plots)@\spxentry{rgb\_overlay()}\spxextra{in module cubespa.plotting.overlay\_plots}}

\begin{fulllineitems}
\phantomsection\label{\detokenize{modules:cubespa.plotting.overlay_plots.rgb_overlay}}
\pysigstartsignatures
\pysiglinewithargsret{\sphinxcode{\sphinxupquote{cubespa.plotting.overlay\_plots.}}\sphinxbfcode{\sphinxupquote{rgb\_overlay}}}{\emph{\DUrole{n}{rgb\_img}}, \emph{\DUrole{n}{overlay\_obj}}, \emph{\DUrole{n}{lims}\DUrole{o}{=}\DUrole{default_value}{None}}, \emph{\DUrole{n}{levels}\DUrole{o}{=}\DUrole{default_value}{None}}, \emph{\DUrole{n}{colors}\DUrole{o}{=}\DUrole{default_value}{None}}, \emph{\DUrole{n}{filename}\DUrole{o}{=}\DUrole{default_value}{None}}, \emph{\DUrole{o}{**}\DUrole{n}{kwargs}}}{}
\pysigstopsignatures
\end{fulllineitems}

\phantomsection\label{\detokenize{modules:module-cubespa.plotting.spectra_plots}}\index{module@\spxentry{module}!cubespa.plotting.spectra\_plots@\spxentry{cubespa.plotting.spectra\_plots}}\index{cubespa.plotting.spectra\_plots@\spxentry{cubespa.plotting.spectra\_plots}!module@\spxentry{module}}\index{multispec\_plot() (in module cubespa.plotting.spectra\_plots)@\spxentry{multispec\_plot()}\spxextra{in module cubespa.plotting.spectra\_plots}}

\begin{fulllineitems}
\phantomsection\label{\detokenize{modules:cubespa.plotting.spectra_plots.multispec_plot}}
\pysigstartsignatures
\pysiglinewithargsret{\sphinxcode{\sphinxupquote{cubespa.plotting.spectra\_plots.}}\sphinxbfcode{\sphinxupquote{multispec\_plot}}}{\emph{\DUrole{n}{cubespa\_obj}}, \emph{\DUrole{n}{aper\_list}}, \emph{\DUrole{n}{spec\_list}}, \emph{\DUrole{o}{**}\DUrole{n}{kwargs}}}{}
\pysigstopsignatures
\end{fulllineitems}

\index{plot\_spectra() (in module cubespa.plotting.spectra\_plots)@\spxentry{plot\_spectra()}\spxextra{in module cubespa.plotting.spectra\_plots}}

\begin{fulllineitems}
\phantomsection\label{\detokenize{modules:cubespa.plotting.spectra_plots.plot_spectra}}
\pysigstartsignatures
\pysiglinewithargsret{\sphinxcode{\sphinxupquote{cubespa.plotting.spectra\_plots.}}\sphinxbfcode{\sphinxupquote{plot\_spectra}}}{\emph{\DUrole{n}{data}}, \emph{\DUrole{n}{aper}}}{}
\pysigstopsignatures
\end{fulllineitems}

\index{spectra\_comparison() (in module cubespa.plotting.spectra\_plots)@\spxentry{spectra\_comparison()}\spxextra{in module cubespa.plotting.spectra\_plots}}

\begin{fulllineitems}
\phantomsection\label{\detokenize{modules:cubespa.plotting.spectra_plots.spectra_comparison}}
\pysigstartsignatures
\pysiglinewithargsret{\sphinxcode{\sphinxupquote{cubespa.plotting.spectra\_plots.}}\sphinxbfcode{\sphinxupquote{spectra\_comparison}}}{\emph{\DUrole{n}{cubecomp}}, \emph{\DUrole{n}{a1}}, \emph{\DUrole{n}{a2}}, \emph{\DUrole{n}{s1}}, \emph{\DUrole{n}{s2}}, \emph{\DUrole{n}{chan\_ranges}\DUrole{o}{=}\DUrole{default_value}{None}}, \emph{\DUrole{o}{**}\DUrole{n}{kwargs}}}{}
\pysigstopsignatures
\end{fulllineitems}

\index{spectra\_plot() (in module cubespa.plotting.spectra\_plots)@\spxentry{spectra\_plot()}\spxextra{in module cubespa.plotting.spectra\_plots}}

\begin{fulllineitems}
\phantomsection\label{\detokenize{modules:cubespa.plotting.spectra_plots.spectra_plot}}
\pysigstartsignatures
\pysiglinewithargsret{\sphinxcode{\sphinxupquote{cubespa.plotting.spectra\_plots.}}\sphinxbfcode{\sphinxupquote{spectra\_plot}}}{\emph{\DUrole{n}{cubespa\_obj}}, \emph{\DUrole{n}{aper}}, \emph{\DUrole{n}{spectrum}}}{}
\pysigstopsignatures\begin{description}
\sphinxlineitem{Create a plot showing both the image with overlaid spectra, as well as the spectrum with}
\sphinxAtStartPar
RMS levels shown.

\end{description}
\begin{quote}\begin{description}
\sphinxlineitem{Parameters}\begin{itemize}
\item {} 
\sphinxAtStartPar
\sphinxstyleliteralstrong{\sphinxupquote{cubespa\_obj}} ({\hyperref[\detokenize{modules:cubespa.cubespa.CubeSPA}]{\sphinxcrossref{\sphinxstyleliteralemphasis{\sphinxupquote{cubespa.CubeSPA}}}}}) \textendash{} CubeSPA object.

\item {} 
\sphinxAtStartPar
\sphinxstyleliteralstrong{\sphinxupquote{aper}} (\sphinxstyleliteralemphasis{\sphinxupquote{photutils.aperture}}) \textendash{} Input aperture, generated using cubespa.spectra

\item {} 
\sphinxAtStartPar
\sphinxstyleliteralstrong{\sphinxupquote{spectrum}} (\sphinxstyleliteralemphasis{\sphinxupquote{\_type\_}}) \textendash{} \_description\_

\end{itemize}

\end{description}\end{quote}

\end{fulllineitems}

\phantomsection\label{\detokenize{modules:module-cubespa.plotting.util_plots}}\index{module@\spxentry{module}!cubespa.plotting.util\_plots@\spxentry{cubespa.plotting.util\_plots}}\index{cubespa.plotting.util\_plots@\spxentry{cubespa.plotting.util\_plots}!module@\spxentry{module}}\index{limit\_plot() (in module cubespa.plotting.util\_plots)@\spxentry{limit\_plot()}\spxextra{in module cubespa.plotting.util\_plots}}

\begin{fulllineitems}
\phantomsection\label{\detokenize{modules:cubespa.plotting.util_plots.limit_plot}}
\pysigstartsignatures
\pysiglinewithargsret{\sphinxcode{\sphinxupquote{cubespa.plotting.util\_plots.}}\sphinxbfcode{\sphinxupquote{limit\_plot}}}{\emph{\DUrole{n}{cubespa\_obj}}}{}
\pysigstopsignatures
\end{fulllineitems}

\index{plot\_bbox() (in module cubespa.plotting.util\_plots)@\spxentry{plot\_bbox()}\spxextra{in module cubespa.plotting.util\_plots}}

\begin{fulllineitems}
\phantomsection\label{\detokenize{modules:cubespa.plotting.util_plots.plot_bbox}}
\pysigstartsignatures
\pysiglinewithargsret{\sphinxcode{\sphinxupquote{cubespa.plotting.util\_plots.}}\sphinxbfcode{\sphinxupquote{plot\_bbox}}}{\emph{\DUrole{n}{cubespa\_obj}}, \emph{\DUrole{n}{lims}}}{}
\pysigstopsignatures
\end{fulllineitems}

\index{plot\_rgb() (in module cubespa.plotting.util\_plots)@\spxentry{plot\_rgb()}\spxextra{in module cubespa.plotting.util\_plots}}

\begin{fulllineitems}
\phantomsection\label{\detokenize{modules:cubespa.plotting.util_plots.plot_rgb}}
\pysigstartsignatures
\pysiglinewithargsret{\sphinxcode{\sphinxupquote{cubespa.plotting.util\_plots.}}\sphinxbfcode{\sphinxupquote{plot\_rgb}}}{\emph{\DUrole{n}{rgb}}, \emph{\DUrole{n}{lims}\DUrole{o}{=}\DUrole{default_value}{None}}, \emph{\DUrole{n}{outname}\DUrole{o}{=}\DUrole{default_value}{None}}}{}
\pysigstopsignatures
\sphinxAtStartPar
Plot an RGB image using matplotlib
\begin{quote}\begin{description}
\sphinxlineitem{Parameters}\begin{itemize}
\item {} 
\sphinxAtStartPar
\sphinxstyleliteralstrong{\sphinxupquote{rgb}} (\sphinxstyleliteralemphasis{\sphinxupquote{nxmx3 array}}) \textendash{} RGB image formatted for matplotlib

\item {} 
\sphinxAtStartPar
\sphinxstyleliteralstrong{\sphinxupquote{lims}} (\sphinxstyleliteralemphasis{\sphinxupquote{arr}}\sphinxstyleliteralemphasis{\sphinxupquote{, }}\sphinxstyleliteralemphasis{\sphinxupquote{optional}}) \textendash{} x and y limits for plotting. Defaults to None.

\item {} 
\sphinxAtStartPar
\sphinxstyleliteralstrong{\sphinxupquote{outname}} (\sphinxstyleliteralemphasis{\sphinxupquote{str}}\sphinxstyleliteralemphasis{\sphinxupquote{, }}\sphinxstyleliteralemphasis{\sphinxupquote{optional}}) \textendash{} Output filename. If not, show plot instead of
save figure. Defaults to None.

\end{itemize}

\end{description}\end{quote}

\end{fulllineitems}

\begin{itemize}
\item {} 
\sphinxAtStartPar
\DUrole{xref,std,std-ref}{genindex}

\item {} 
\sphinxAtStartPar
\DUrole{xref,std,std-ref}{modindex}

\item {} 
\sphinxAtStartPar
\DUrole{xref,std,std-ref}{search}

\end{itemize}


\renewcommand{\indexname}{Python Module Index}
\begin{sphinxtheindex}
\let\bigletter\sphinxstyleindexlettergroup
\bigletter{c}
\item\relax\sphinxstyleindexentry{cubespa.cubespa}\sphinxstyleindexpageref{modules:\detokenize{module-cubespa.cubespa}}
\item\relax\sphinxstyleindexentry{cubespa.data}\sphinxstyleindexpageref{modules:\detokenize{module-cubespa.data}}
\item\relax\sphinxstyleindexentry{cubespa.plotting.channel\_map\_plots}\sphinxstyleindexpageref{modules:\detokenize{module-cubespa.plotting.channel_map_plots}}
\item\relax\sphinxstyleindexentry{cubespa.plotting.mommap\_plots}\sphinxstyleindexpageref{momentmaps:\detokenize{module-cubespa.plotting.mommap_plots}}
\item\relax\sphinxstyleindexentry{cubespa.plotting.overlay\_plots}\sphinxstyleindexpageref{modules:\detokenize{module-cubespa.plotting.overlay_plots}}
\item\relax\sphinxstyleindexentry{cubespa.plotting.spectra\_plots}\sphinxstyleindexpageref{spectra:\detokenize{module-cubespa.plotting.spectra_plots}}
\item\relax\sphinxstyleindexentry{cubespa.plotting.util\_plots}\sphinxstyleindexpageref{modules:\detokenize{module-cubespa.plotting.util_plots}}
\item\relax\sphinxstyleindexentry{cubespa.spectra}\sphinxstyleindexpageref{spectra:\detokenize{module-cubespa.spectra}}
\item\relax\sphinxstyleindexentry{cubespa.utils}\sphinxstyleindexpageref{modules:\detokenize{module-cubespa.utils}}
\end{sphinxtheindex}

\renewcommand{\indexname}{Index}
\printindex
\end{document}